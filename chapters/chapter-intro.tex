% chapter1



\chapter{Introduction}
\label{chap:introduction}
\index{Introduction@\emph{Introduction}}%

Quantum computation is an exciting new field of research for a number 
of reasons\cite{Preskill:98}:  
\begin{itemize}
\item Quantum computers can solve some hard problems.
\item Quantum errors can be corrected.
\item Quantum hardware can be constructed.
\end{itemize}
The field lies on the boundary between Physics, Mathematics, 
and Computer Science.  It is developing at a rapid pace because
theoretical and experimental researchers are working closely together
in quite a unified effort.

The age of single photon and single atom experiments is upon us. 
The experimental techniques for controlling quantum computations 
either already exist or are expected to exist in the near future.
Researchers on the theoretical side are contributing
things like quantum error correcting codes that can be used to 
increase the quantum computer's tolerance for noise, as well as the
algorithms used to solve various problems on a quantum computer.

What is it all about?  Why all the hype?
To see this, it is useful to look at some of the differences 
between a classical and quantum computer\dots to examine
some of the limitations of the classical computers in use
today.

Cosider the representation and evaluation of a function on a classical
computer.  Take, for instance, some function $f\colon\Z\to\Z$ where
\begin{equation}
x\mapsto f(x).
\label{e:funcs}
\end{equation}
Now, an integer is represented on a classical computer, in base-two, by a 
finite set of binary digits or \emph{bits}.
\index{bits}
If the computer represents
integers by say $n$ bits,
\begin{equation}
\begin{split}
x =& \underset{n\text{ bits}}{\underbrace{0110010\ldots 0}}\\
=& b_{n-1}b_{n-2}\ldots b_1b_0\\
=&\quad\text{(in base 10)}\quad\sum_{i=0}^{n-1} b_i2^i,
\end{split}
\end{equation}
then only $2^n$ different integers can be represented.
The function $f\colon\Z\to\Z$ really becomes $f\colon\Z_{2^n}\to\Z_{2^n}$.

Next, consider some very basic questions about the function $f$.
Is it constant? %\ie, does $f(x)=c$ for all $x\in\Z_{2^n}$?
Is it monotonic?  Is it periodic?  If so, find the period.
How can these questions be answered on a classical computer?
Well, the only thing to do is evaluate the function for \emph{all} 
inputs.  If the function is constant for all inputs, it is constant.
In general, this is expensive to compute\dots prohibitively so.  
One might truthfully
say that this sort of menial task is exactly what computers excel
at and so, who cares?  What's the big deal?  Well, consider the 
case of determining the periodicity of a function.  This is used,
for instance, in certain factoring algorithms that allow a classical
computer to factor a large integer into it's component prime 
factors.\footnote{This is covered in detail in chapter 
\ref{chap:quantumAlgorithms}.}
Again, in order to tackle this problem, the function must be evaluated
for \emph{all} inputs.  It is the difficulty of this very task
that lies behind the security of most cryptosystems in use today.
These methods rely upon the fact that it takes a \emph{long} time 
to determine the periodicity of a function and hence,
the factors of a large integer.  

The question to now ask is\dots can we do better?  The answer is a
resounding yes!  A \emph{quantum} computer can answer some of these
questions with considerably less work.  This is what has caused all
of the hype.  They are not better, or even good, at everything.  
However, it has been shown that if they could be built,
quantum computers can solve certain problems \emph{much} faster that
their classical cousins.

In order to see what quantum computers are all about, first consider 
the representation of, say, integers on a quantum computer.
The {\sl bits} of a classical computer have a quantum mechanical analogue,
the quantum bit or \emph{qubit}.\index{qubit}
A quantum bit is some two--state system.
%\footnote{Exactly what kind
Exactly what kind
of two--state system is not specified here.  There are many
different implementations of qubits currently being investigated
in laboratories
around the world, and it is not clear which physical implementation
will work best.  
It should be noted that a major issue in determining which physical system 
to use as a quantum computer is the noise inherent to that system.  
Research such as this dissertation is exactly the type of work necessary 
to determine the tolerances that the physical implementations must be held 
to, and ultimately, to determine which physical systems will be workable 
in an actual quantum computer.
Thinking of a qubit as a simple spin--$\frac{1}{2}$ system
is probably best for now.
A bit is represented by either spin up or down
\begin{equation}
\begin{split}
0\qquad\mapsto\qquad&\ket{0}=\ket{\downarrow}\\
1\qquad\mapsto\qquad&\ket{1}=\ket{\uparrow}\\
\end{split}
\end{equation}
An integer represented on a classical computer by a set of $n$ bits
\begin{equation}
x = (01100011\ldots0)
\end{equation}
can be represented as a set of $n$ qubits
\begin{equation}
\begin{split}
\ket{x} =& \ket{01100011\ldots0}\\
  =& \bigotimes_{i=0}^{n-1} \ket{b_i}.
\end{split}
\end{equation}

Now, consider the representation and evaluation of functions such as 
(\ref{e:funcs}) on a \emph{quantum} computer.
If $x$ becomes $\ket{x}$, and then the value of the function is just
another integer represented on the computer by $\ket{f(x)}$.
How can $f(x)$ be evaluated?  How can the state $\ket{f(x)}$ be
obtained from the state $\ket{x}$?  Well, closed quantum systems 
evolve by unitary transformations, so an operator such as
\begin{equation}
{\bf U_f}\ket{x} = \ket{f(x)}
\end{equation}
just might be a good start.
Actually, the unitarity of the above operator is dependent upon
properties of the function $f$.
A better approach is to take the operator
\begin{equation}
{\bf U_f}\ket{x}\ket{0} = \ket{x}\ket{f(x)},
\end{equation}
or really
\begin{equation}
{\bf U_f}\ket{x}\ket{a} = \ket{x}\ket{f(x)\oplus a}\qquad\text{(any $a$)},
\end{equation}
which is unitary no matter the function used.
Exactly how such unitary operators can be 
constructed is covered in detail later in 
chapter \ref{chap:quantumComputation}.

Now, given such a unitary operator, the function $f$ can be 
evaluated for some input $\ket{x}$ on a quantum computer.
We can evaluate a function for some input.
Big deal.  Is this all just a fancier way of writing the same
classical problem?  
The same constraints seemingly apply. \ie, determining if the
function is constant still seems to require evaluations of the
function for all inputs.  Well\dots no.

The classical evaluation of a function relies upon gates that
can act on a bit with a definite value, 0 or 1.  The quantum
function evaluation is accomplished by unitary transformations
that can act, not only on states such as $\ket{0}$ or
$\ket{1}$, but on superpositions of states such as 
$a\ket{0}+b\ket{1}$.  This is the crucial difference -- that 
quantum computations can act on superpositions of states.  
This fact allows the quantum computer to take advantage of a 
sort of pseudo--parallelism to beat out the classical computer
in solving certain types of problems.

Consider the example of finding the period of a periodic function 
$f$.\footnote{Periodic here refers to some periodicity besides the
obvious one due to the computer's finite representation of integers
as elements of the ring $\Z_{2^n}$.}
Instead of taking 
\begin{equation}
{\bf U_f}\ket{x}\ket{0} = \ket{x}\ket{f(x)}
\end{equation}
for all inputs $\ket{x}$, consider acting on the state
\begin{equation}
\frac{1}{\sqrt{2^n}}\sum_{x=0}^{2^n-1}\ket{x}\ket{0}
\end{equation}
which is in an equally weighted superposition of all input
states $\ket{x}$.  The function evaluation operator
\begin{equation}
{\bf U_f} \left\lbrack
\frac{1}{\sqrt{2^n}}\sum_{x=0}^{2^n-1}\ket{x}\ket{0}
\right\rbrack
= 
\frac{1}{\sqrt{2^n}}\sum_{x=0}^{2^n-1}\ket{x}\ket{f(x)}
\end{equation}
evaluates the function simultaneously for \emph{all} inputs!
This is a powerful idea.  The resulting state contains all
information about the function.  Of course, we don't necessarily
have {\sl access} to all of this information, but the right 
measurements can allow certain questions to be answered.
The rest of the ``algorithm'' to find the period consists
of a measurement of the second ket
\begin{equation}
\bigl(
{\bf 1}\otimes\ketbra{c}{c}
\bigr)
\sum_{x=0}^{2^n-1}\ket{x}\ket{f(x)}
= \sum_{\lbrace x|f(x)=c\rbrace} \ket{x}\ket{c}
\end{equation}
which
collapses the superposition of all $\ket{x}$ into a 
superposition of just those
where $f(x)=c$.  The periodicity of the function $f$ is still
contained within this superposition and can be extracted via
a Fourier transform.
This entire process takes {\sl considerably} less time than the classical
version.

\subsubsection{This Dissertation} 

There are several quantum algorithms to solve various problems
on a quantum computer that exhibit various degrees of 
speedup over their classical counterparts.  
The quantum search algorithm is one such algorithm that is
interesting for several reasons: it can easily be adapted to
solve a host of other problems, it exhibits a speedup over a
classical search algorithm, it is particularly simple to 
describe, and it can be described dynamically.
This is often the first algorithm that experimentalists attempt
to implement, and so it is a natural candidate for a 
robustness analysis.

My current research concentrates on a dynamical model of this
quantum search algorithm.  The hope is that, in future work,
a dynamical approach can be taken for quantum alorithms in 
general.  The vast arsenal of tools from the discipline of 
dynamical systems could then be brought to bear upon the 
analysis of the stability of various quantum algorithms.

For this dissertation, a faithful numerical model of the search
algorithm is developed.  Noise is then added
in a generic way that will encompass any source and type of 
physical noise.
The goal is to determine the robustness of the bare algorithm
in the presence of various types of noise.

This dissertation is organized as follows:  Chapters \ref{chap:quantumComputation}
and \ref{chap:quantumAlgorithms} contain background information on quantum computation
and quantum algorithms.
Chapter \ref{chap:quantumGeometry} describes the
Fubini--Study metric and relates this to a statistical distance in Hilbert space.
Chapter \ref{chap:dynamics} applies the Fubini--Study  metric to a dynamical 
model of the quantum search algorithm.
It introduces the
necessary techniques for numerical stability analysis that are then applied to
the model of the algorithm in the presence of noise. 
Results are presented in chapter \ref{chap:conclusion}, with
data included in appendix \ref{chap:data}.
Appendix \ref{chap:code} discusses details of the numerical model and the 
tools used in this analysis. 


