% chapter1



\chapter{Introduction}
\label{chap:introduction}
\index{Introduction@\emph{Introduction}}%

This is some text...

Richard Feynman blahblah.
In 19??, he wrote a pair of papers about the apparent 
difficulty of simulating physical systems on a computer.  
To simulate even simple physics requires quite a lot
of computing power.
This kind of simulation requires a program whose time
or space required to run scales exponentially with the
size of input.

The current ideas of quantum computation involve reversing
the argument. \ie, Evolutions of a quantum mechanical 
system are equivalent to (somewhat larger) evolutions of a
``classical'' computer.  Instead of using the monstrous 
classical computation to simulate the physical evolution,
use the simple physical evolution to efficiently obtain 
answers to difficult classical computations.

In this work, Feynman also introduced the concept of reversible 
computation \mmm{really?}.  


%\notde{ldfk}

This dissertation is organized as follows:  Chapters \ref{chap:quantumComputation}
and \ref{chap:quantumAlgorithms} contain background information on quantum computation
and quantum algorithms.
%These chapters are essentially from\cite{Preskill:98,Steane:98,Braunstein:99,Shor:00}
%and are included for completeness.  
Chapter \ref{chap:quantumGeometry} describes the
Fubini--Study metric and relates this to a statistical distance in Hilbert space.

The meat of the current work lies in the remaining chapters.  
Chapter \ref{chap:dynamics} applies the Fubini--Study  metric to a dynamical 
model of a particular quantum algorithm, namely Grover's quantum search algorithm.  
Chapter \ref{chap:dynamicalStability} introduces the
necessary techniques for numerical stability analysis that are then applied to
this model of Grover's algorithm in the presence of noise. 
Finally, chapter \ref{chap:code} discusses aspects of the numerical model and the 
tools used in this analysis. 


