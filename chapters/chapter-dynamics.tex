
\chapter{A Dynamical Approach to Quantum Algorithms}
\label{chap:dynamics}
\index{Quantum Algorithms@\emph{Quantum Algorithms}!Dynamical Approach@\emph{Dynamical Approach}}%

%--------------------------------------------------------------------------------   
%--------------------------------------------------------------------------------   
\section{Grover}

%--------------------------------------------------------------------------------   
\subsection{Sines and Cosines}
\label{ssec:sincos}

Grover's algorithm was introduced and discussed at great length in
chapter (\ref{chap:quantumAlgorithms}).  There, it was mentioned that
Grover's algorithm can be seen as a rotation of states in Hilbert
space.  This section will establish this result a little more rigorously.

Consider the action of Grover's algorithm as described in the end of
section (\ref{sec:grover}).  This will describe a search for one 
out of a list of $N$ items.  At any given step in the process, the
quantum register is in the state
\begin{equation}
\ket{\Psi} = k\ket{\omega} + \sum_{j\in\omega_\perp}l\ket{j},
\end{equation}
for some $k$ and $l$, which, at say the $i$-th iteration, 
look like
\begin{equation}
\begin{split}
k_{i+1} =& \frac{N-2}{N}k_i + \frac{2(N-1)}{N}l_i\\
l_{i+1} =& -\frac{2}{N}k_i + \frac{N-2}{N}l_i.
\end{split}
\label{e:discsincos}
\end{equation}

\begin{prop}
The coefficients $k$ and $l$ obey
\begin{equation}
\begin{split}
k_i =& \sin\left[\left(2i+1\right)\theta\right]\\
l_i =& \frac{1}{\sqrt{N-1}}\cos\left[\left(2i+1\right)\theta\right]
\end{split}
\label{e:sincos}
\end{equation}
for $\sin\theta = \frac{1}{\sqrt{N}}$.
\end{prop}

\begin{proof}
(By induction)  
Starting at step zero with an equally weighted 
superposition of all states
\begin{equation}
\ket{\Psi_0} = \frac{1}{\sqrt{N}}\ket{\omega} 
                + \sum_{j\in\omega_\perp}\frac{1}{\sqrt{N}}\ket{j}
\end{equation}
implies $k_0 = l_0 = \frac{1}{\sqrt{N}}$.
Inserting $i=0$ into (\ref{e:sincos}), you get
\begin{equation}
k_0 = \sin\left(\theta\right) = \frac{1}{\sqrt{N}}
\end{equation}
and 
\begin{equation}
\begin{split}
l_0 =& \frac{1}{\sqrt{N-1}}\cos\left(\theta\right)\\
    =& \frac{1}{\sqrt{N-1}}\sqrt{ 1 - \sin^2\left(\theta\right) }\\
    =& \frac{1}{\sqrt{N-1}}\sqrt{ 1 - \frac{1}{N} }
    = \frac{1}{\sqrt{N}}.
\end{split}
\end{equation}
The proposition holds for $i=0$.

Now, consider the $(i+1)$--th iteration
\begin{equation}
\begin{split}
&\begin{cases}
k_{i+1} =& \sin\left[\left(2(i+1)+1\right)\theta\right]\\
l_{i+1} =& \frac{1}{\sqrt{N-1}}\cos\left[\left(2(i+1)+1\right)\theta\right]
\end{cases}\\
&\begin{cases}
k_{i+1} =& \sin\left[\left(2i+1\right)\theta+2\theta\right]\\
l_{i+1} =& \frac{1}{\sqrt{N-1}}\cos\left[\left(2i+1\right)+2\theta\right]
\end{cases}\\
&\begin{cases}
k_{i+1} =& \sin\left[\left(2i+1\right)\theta\right]
           \cos\left(2\theta\right)
          +\cos\left[\left(2i+1\right)\theta\right]
           \sin\left(2\theta\right) \\
l_{i+1} =& \frac{1}{\sqrt{N-1}}\left\lbrace
                \cos\left[\left(2i+1\right)\theta\right]
                \cos\left(2\theta\right)
               -\sin\left[\left(2i+1\right)\theta\right]
                \sin\left(2\theta\right) 
           \right\rbrace.
\end{cases}
\end{split}
\end{equation}
By the induction hypothesis, 
\begin{equation}
\begin{cases}
k_{i+1} =& k_i\cos\left(2\theta\right)
          +\sqrt{N-1}\;l_i\sin\left(2\theta\right) \\
l_{i+1} =& \frac{1}{\sqrt{N-1}}\left\lbrace
                \sqrt{N-1}\;l_i\cos\left(2\theta\right)
               - k_i\sin\left(2\theta\right) 
           \right\rbrace,
\end{cases}
\end{equation}
and since
\begin{equation}
\sin\left(\theta\right) = \frac{1}{\sqrt{N}}
\quad\Rightarrow\quad
\cos\left(2\theta\right) = \frac{N-2}{N}
\quad\text{and}\quad
\sin\left(2\theta\right) = \frac{2\sqrt{N-1}}{N},
\end{equation}
then
\begin{equation}
\begin{cases}
k_{i+1} =& k_i\frac{N-2}{N} + l_i\frac{2(N-1)}{N}\\
l_{i+1} =& l_i\frac{N-2}{N} -k_i\frac{2}{N}. 
\end{cases}
\end{equation}
Since this is (\ref{e:discsincos}) as expected, $j\Rightarrow j+1$.
\end{proof}


\mmm{Trotter formula!}

\mmm{Biggest problem still: Why is Grover a geodesic?}

%--------------------------------------------------------------------------------   
\subsection{Grover is a Geodesic}

Section (\ref{ssec:sincos}) expressed Grover's algorithm in terms of
trigonometric functions of discrete quantities.  The coefficients of
the state
\begin{equation}
\ket{\Psi} = k\ket{\omega} + \sum_{j\in\omega_\perp}l\ket{j}
\end{equation}
evolve according to
\begin{equation}
\begin{split}
k_i =& \sin\left[\left(2i+1\right)\theta\right]\\
l_i =& \frac{1}{\sqrt{N-1}}\cos\left[\left(2i+1\right)\theta\right].
\end{split}
\end{equation}
This evolution can be viewed as a discrete path in state space
\mmm{figure}.

\begin{prop}
The continuous path connecting these points is a geodesic
of the Fubini--Study metric.
\end{prop}

\begin{proof}
Duh!
\end{proof}





%--------------------------------------------------------------------------------   
%--------------------------------------------------------------------------------   
\section{Dynamical Maps}

\mmm{This chapter should introduce the numerical model and show some baseline
results before noise or stability is even considered}
