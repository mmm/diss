
\chapter{A Dynamical Approach to Quantum Algorithms}
\label{chap:dynamics}
\index{Quantum Algorithms@\emph{Quantum Algorithms}!Dynamical Approach@\emph{Dynamical Approach}}%

%--------------------------------------------------------------------------------   
%--------------------------------------------------------------------------------   
\section{Grover}

\label{sec:sincos}

Grover's algorithm was introduced and discussed at great length in
chapter (\ref{chap:quantumAlgorithms}).  There, it was mentioned that
Grover's algorithm can be seen as a rotation of states in Hilbert
space.  This section will establish this result a little more rigorously 
and show how it is used in an analysis of the complexity of the algorithm.

Consider the discrete action of Grover's algorithm as described in the end of
section (\ref{sec:grover}).  This will describe a search for \emph{one},
state $\ket\omega$,
out of a list of $N$ states.  At any given step in the process, the
quantum database is in the state
\begin{equation}
\ket{\Psi} = k\ket{\omega} + \sum_{j\in\omega_\perp}l\ket{j},
\end{equation}
for some $k$ and $l$, which, at say the $i$-th iteration, 
look like
\begin{equation}
\begin{split}
k_{i+1} =& \frac{N-2}{N}k_i + \frac{2(N-1)}{N}l_i\\
l_{i+1} =& -\frac{2}{N}k_i + \frac{N-2}{N}l_i.
\end{split}
\label{e:discsincos}
\end{equation}

\begin{prop}
\label{prop:sincos}
The coefficients $k$ and $l$ in (\ref{e:discsincos}) obey
\begin{equation}
\begin{split}
k_i =& \sin\left[\left(2i+1\right)\theta\right]\\
l_i =& \frac{1}{\sqrt{N-1}}\cos\left[\left(2i+1\right)\theta\right]
\end{split}
\label{e:sincos}
\end{equation}
for $\sin\theta = \frac{1}{\sqrt{N}}$.
\end{prop}

\begin{proof}
(By induction)  
Starting at step zero with an equally weighted 
superposition of all states
\begin{equation}
\ket{\Psi_0} = \frac{1}{\sqrt{N}}\ket{\omega} 
                + \sum_{j\in\omega_\perp}\frac{1}{\sqrt{N}}\ket{j}
\end{equation}
implies $k_0 = l_0 = \frac{1}{\sqrt{N}}$.
Inserting $i=0$ into (\ref{e:sincos}), you get
\begin{equation}
k_0 = \sin\left(\theta\right) = \frac{1}{\sqrt{N}}
\end{equation}
and 
\begin{equation}
\begin{split}
l_0 =& \frac{1}{\sqrt{N-1}}\cos\left(\theta\right)\\
    =& \frac{1}{\sqrt{N-1}}\sqrt{ 1 - \sin^2\left(\theta\right) }\\
    =& \frac{1}{\sqrt{N-1}}\sqrt{ 1 - \frac{1}{N} }
    = \frac{1}{\sqrt{N}}.
\end{split}
\end{equation}
So, the proposition holds for $i=0$.

Now, consider the $(i+1)$--th iteration
\begin{equation}
\begin{split}
&\begin{cases}
k_{i+1} =& \sin\left[\left(2(i+1)+1\right)\theta\right]\\
l_{i+1} =& \frac{1}{\sqrt{N-1}}\cos\left[\left(2(i+1)+1\right)\theta\right]
\end{cases}\\
&\begin{cases}
k_{i+1} =& \sin\left[\left(2i+1\right)\theta+2\theta\right]\\
l_{i+1} =& \frac{1}{\sqrt{N-1}}\cos\left[\left(2i+1\right)+2\theta\right]
\end{cases}\\
&\begin{cases}
k_{i+1} =& \sin\left[\left(2i+1\right)\theta\right]
           \cos\left(2\theta\right)
          +\cos\left[\left(2i+1\right)\theta\right]
           \sin\left(2\theta\right) \\
l_{i+1} =& \frac{1}{\sqrt{N-1}}\left\lbrace
                \cos\left[\left(2i+1\right)\theta\right]
                \cos\left(2\theta\right)
               -\sin\left[\left(2i+1\right)\theta\right]
                \sin\left(2\theta\right) 
           \right\rbrace.
\end{cases}
\end{split}
\end{equation}
By the induction hypothesis, 
\begin{equation}
\begin{cases}
k_{i+1} =& k_i\cos\left(2\theta\right)
          +\sqrt{N-1}\;l_i\sin\left(2\theta\right) \\
l_{i+1} =& \frac{1}{\sqrt{N-1}}\left\lbrace
                \sqrt{N-1}\;l_i\cos\left(2\theta\right)
               - k_i\sin\left(2\theta\right) 
           \right\rbrace,
\end{cases}
\end{equation}
and since
\begin{equation}
\sin\left(\theta\right) = \frac{1}{\sqrt{N}}
\quad\Rightarrow\quad
\cos\left(2\theta\right) = \frac{N-2}{N}
\quad\text{and}\quad
\sin\left(2\theta\right) = \frac{2\sqrt{N-1}}{N},
\end{equation}
then
\begin{equation}
\begin{cases}
k_{i+1} =& k_i\frac{N-2}{N} + l_i\frac{2(N-1)}{N}\\
l_{i+1} =& l_i\frac{N-2}{N} -k_i\frac{2}{N}. 
\end{cases}
\end{equation}
Since this is (\ref{e:discsincos}) as expected, $j\Rightarrow j+1$.
\end{proof}
Now, the cyclical nature of this evolution means that the
algorithm must be stopped manually when the probability 
for success will be high.
Proposition \ref{prop:sincos} allows the ``stopping time''
for the algorithm to be easily estimated.  The expressions 
for $k$ and $l$ in (\ref{e:sincos}) take extreme values 
when
\begin{equation}
(2i+1)\theta = \frac{\pi}{2},
\end{equation}
and so, along with the fact that $\theta\sim\frac{1}{\sqrt{N}}$ 
for large $N$, 
the algorithm must be stopped in about 
$\lfloor\frac{\pi}{4}\sqrt{N}\rfloor$ iterations.
For a more detailed analysis of stopping time for Grover's
algorithm (and multisearch too) see \cite{Grover:96a,Boyer/Brassard/Hoyer/Tapp:96}.
It should also be noted that
Grover's algorithm has been 
shown\cite{Bennet/Brassard/Bernstein/Vazirani:97,Zalka:97}
to be optimal.  \ie, there are no other search algorithms
more efficient than Grover.



%%--------------------------------------------------------------------------------   
%\subsection{Grover is Optimal}
%
%\mmm{add Zalka stuff and multisearch stopping time (from Boyer et al) here}

%--------------------------------------------------------------------------------   
\subsection{Grover is a Geodesic}

Earlier in this chapter Grover's algorithm was expressed in terms of
discrete recurrence relations.  In the algorithm, the coefficients of
the state
\begin{equation}
\ket{\Psi} = k\ket{\omega} + \sum_{j\in\omega_\perp}l\ket{j}
\end{equation}
evolve according to
\begin{equation}
\begin{split}
k_i =& \sin\left[\left(2i+1\right)\theta\right]\\
l_i =& \frac{1}{\sqrt{N-1}}\cos\left[\left(2i+1\right)\theta\right].
\end{split}
\end{equation}
This evolution can be viewed as a discrete path in state space
as shown in figure (\ref{fig:discreteGrover}).
\begin{figure}[h]
\begin{center}
\begin{picture}(200,150)
    \thicklines
    \qbezier(0,70)(40,150)(120,100)
    \qbezier(120,100)(150,80)(180,85)
    \qbezier(0,70)(-40,-50)(140,0)
    \qbezier(180,85)(270,90)(200,30)
    \qbezier(200,30)(170,10)(140,0)
    \put(20,20){$\ket{\Psi}$}
    \put(110,70){$\ket{\omega}$}
    \qbezier[15](25,35)(70,70)(175,60)
\end{picture}
\caption{Grover's algorithm is a discrete path in state space 
starting from an equally--weighted superposition $\ket\Psi$ and 
continuing on through the desired state $\ket\omega$.  Note that
the algorithm keeps going and must be stopped at the anticipated
running time (When the algorithm will be near the desired state).}
\label{fig:discreteGrover}
\end{center}
\end{figure}

Now, consider the continuous path on the space of states
given by
\begin{equation}
\gamma\colon \R\to\CP^N\colon t\mapsto 
\begin{pmatrix}
    \sin(t)\\
    \frac{1}{\sqrt{N-1}}\cos(t)\\
    \vdots\\
    \frac{1}{\sqrt{N-1}}\cos(t)
\end{pmatrix}.
\label{e:gamma}
\end{equation}
The discrete points generated by Grover's algorithm 
{\sl lie upon this path}.  This path can then be used
%as a {\sl continuous} approximation to Grover's algorithm
to imbed Grover's algorithm
(figure \ref{fig:contGrover}).
\begin{figure}[h]
\begin{center}
\begin{picture}(200,150)
    \thicklines
    \qbezier(0,70)(40,150)(120,100)
    \qbezier(120,100)(150,80)(180,85)
    \qbezier(0,70)(-40,-50)(140,0)
    \qbezier(180,85)(270,90)(200,30)
    \qbezier(200,30)(170,10)(140,0)
    \put(20,20){$\ket{\Psi}$}
    \put(110,70){$\ket{\omega}$}
    \qbezier(25,35)(70,70)(175,60)
\end{picture}
\caption{Grover's algorithm can be embedded in a continuous
path in state space.  Note that this path is a geodesic of the Fubini--Study
metric.}
\label{fig:contGrover}
\end{center}
\end{figure}

\begin{prop}
The path $\gamma$, as defined in (\ref{e:gamma}),
is a geodesic of the Fubini--Study metric on $\CP^N$.
\end{prop}

\begin{proof}
%In homogeneous coordinates, $\gamma$ is seen to be
%a geodesic of the $(2N+1)$--sphere.
%Since $\CP^N$ can be written as 
%\begin{equation}
%S^1\to S^{2N+1}\to \CP^N,
%\end{equation}
%then the geodesic projects down onto a geodesic of the base.
%\mmm{grin}
In inhomogeneous coordinates, $\gamma$ becomes
\begin{equation}
z_i = \frac{1}{\sqrt{N-1}}\cot(t)
\end{equation}
for all $i$.  A little sweat will show that these satisfy 
the equations of motion (\ref{e:detailedGeodesics}) for a geodesic
of the Fubini--Study metric on $\CP^N$.
\end{proof}


%\begin{prop}
%$\gamma$ coincides with the large database
%limit of Grover's algorithm.
%\end{prop}
%
%\begin{proof}
%Duh!
%\end{proof}
Note that $\gamma$ coincides with the large database
limit of Grover's algorithm.  This can easily be seen from
the large $N$ limit of (\ref{e:sincos}).

%--------------------------------------------------------------------------------   
\subsection{Is there a Hamiltonian?}

The diffusion transformation in Grover's algorithm,
as defined in section (\ref{ssec:diffusion}), looks like
\begin{equation}
{\bf U_s} =
\begin{pmatrix}
    \frac{2-N}{N} & \frac{2}{N} & \cdots & \frac{2}{N} \\
    \frac{2}{N}   & \frac{2-N}{N} & &  \\
    \vdots &  & \ddots & \vdots \\
    \frac{2}{N}   & \cdots && \frac{2-N}{N} 
\end{pmatrix}.
\end{equation}
With a little work, this can be rewritten as
\begin{equation}
{\bf U_s} =
- \exp \left\lbrace  i\frac{\pi}{N}
                \begin{pmatrix}
                    1      &        & \cdots &      & 1 \\
                    \\
                    \vdots &        & \ddots &      & \vdots \\
                    \\
                    1      &        & \cdots &      & 1
                \end{pmatrix}
\right\rbrace.
\end{equation}
For the purpose of argument, assume that the location (index) of
the target state is known ahead of time.  In this case, the
phase shift in Grover's algorithm (\ref{e:oracle}) can be written as
\begin{equation}
{\bf U_\omega} = 
\exp \left\lbrace  i\pi
                \begin{pmatrix}
                    0       &        &        &        &  \\
                            & \ddots &        &        &  \\
                            &        & 1      &        &  \\
                            &        &        & \ddots &  \\
                            &        &        &        & 0
                \end{pmatrix}
\right\rbrace,
\end{equation}
and so the entire Grover iteration can be written
\begin{equation}
\begin{split}
{\bf U}_{\text{Grover}} =& {\bf U}_s {\bf U}_\omega\\ 
=& - \exp \left\lbrace  i\frac{\pi}{N}
                \begin{pmatrix}
                    1      &        & \cdots &      & 1 \\
                    \\
                    \vdots &        & \ddots &      & \vdots \\
                    \\
                    1      &        & \cdots &      & 1
                \end{pmatrix}
\right\rbrace
\exp \left\lbrace  i\pi
                \begin{pmatrix}
                    0       &        &        &        &  \\
                            & \ddots &        &        &  \\
                            &        & 1      &        &  \\
                            &        &        & \ddots &  \\
                            &        &        &        & 0
                \end{pmatrix}
\right\rbrace.
\end{split}
\end{equation}
This should help in finding a Hamiltonian that generates 
the Grover evolution.  
This might prove useful in an analytical stability analysis
of Grover's algorithm, but this has yet to be worked out.



%--------------------------------------------------------------------------------   
%--------------------------------------------------------------------------------   
\section{Dynamical Stability}
\label{sec:dynamicalStability}
\index{Dynamical Stability@\emph{Dynamical Stability}}%


The primary model used in this dissertation describes 
Grover's algorithm as a geodesic of the Fubini--Study metric 
as discussed in the previous section.
A numerical model of the algorithm is developed to determine
how the algorithm behaves in the presence of noise.

%%--------------------------------------------------------------------------------   
%\subsection{Linear Stability Analysis}
%
%Inspection of the geodesic equations of motion 
%(\ref{e:detailedGeodesics}) yields a fixed point about the 
%origin.  No other fixed points are expected. 
%The Jacobian of the evolution vanishes near the origin, and
%so the equations of motion of the linearized problem become
%the trivial geodesic equations for a flat metric.
%This result is not particularly useful in characterizing how 
%the algorithm will behave, at finite step size, in a noisy 
%environment.  
%A method of stability analysis
%is therefore needed that does not rely upon a linearization
%of the problem.  This gets quite difficult to do in general,
%so a numerical approach is warranted.
%
%--------------------------------------------------------------------------------   
\subsection{A Numerical Model of Grover's Algorithm}

A database consisting of $N=2^n$ items, is represented (using
$n$ qubits) as an $N$--state system whose states are given as
points in $\CP^{N-1}$.  The evolution of the algorithm is then 
modelled by the numerical integration of the geodesic equations 
of motion (\ref{e:detailedGeodesics}) on this space.
Appropriate initial conditions are supplied to mimic the behaviour of 
the actual algorithm, which, at this stage, involves only pure 
states.  The unitary evolution of the algorithm simply remains 
on the shell of pure states.
%\mmm{show baseline results here}
%\mmm{pseudotime relates to number of Grover iterations}


%--------------------------------------------------------------------------------   
\subsection{Noise}

The time evolution of a quantum mechanical system is most easily
described by unitary evolution.
Unfortunately (fortunately?), physical systems do not exist in 
isolation.  There
are always interactions with either the environment or other 
physical systems that cause a quantum mechanical system to evolve 
by possibly more complicated means than these simple unitary 
transformations.

A somewhat more general approach to quantum evolution is to 
consider dynamical evolution of a quantum mechanical system
described as simply a map, a \emph{dynamical map}\cite{Sudarshan:??}
of density matrices to density matrices.
%\footnote{Density matrices are needed because\dots}
Notice that an even more general approach is to look at maps of 
density matrices, where the image itself doesn't necessarily have to
be a physical density matrix.  It could be simply a subsystem
of a physical system, and hence it's trace could be less than
one.  For purposes of this dissertation, only dynamical maps that
take physical density matrices to physical density matrices are
considered.

The numerical simulation of Grover's algorithm described 
in the previous section integrates the geodesic equations 
of motion.  This can describe not only the evolution of pure states, 
but more generally, the evolution of density matrices.  
At every time step,
this numerical integration can then be viewed as simply a dynamical
map taking density matrices to density matrices.
\footnote{Really, this particular mapping maps derivatives as well, and
can be thought of as a mapping of the state in a quantum phase 
space.}
%\mmm{maybe example with simple problem as dynamical map...add noise (physically motivated noise)}

Now, any noise introduced into the dynamical evolution of a 
quantum mechanical system must be described within the context 
of dynamical maps.   This approach encompasses any kind of noise
that can be introduced, regardless of the source or if the noise
is Hamiltonian or non-Hamiltonian.

In order to faithfully bring noise into the numerical model of Grover's 
algorithm, mixed states must be represented in that model
(see section \ref{ssec:mixedstate}).  
Mixed state density matrices can be written as
\begin{equation}
\rho = \sum\lambda_i\rho_i,
\end{equation}
where the $\rho_i$ are pure state density matrices, and $i$ can run 
to the dimension of the component pure 
states ($N=2^{\text{Number of Qubits}}$ for the present model). 

Without noise,
the algorithm happily evolves along the shell of pure states.
When noise is added in, the state will be taken off of the shell 
of pure states and ``in'' via a slight mixture of other pure states.
These states are chosen orthogonal to the direction of the target
state in the search, which result in the state moving in a physically
realistic (inward) direction throughout the simulation.

The simulation simultaneously runs two states that start on the shell
as pure states\dots one with noise, one without.
At several times during the run, the Bures distance between the
two states can be calculated (\ref{sec:BuresData}), and the
probability of measuring the target state for each displayed
(figure \ref{fig:humps}).  Data for several runs can be collected
and compiled to give information about the stability of the algorithm.


%%--------------------------------------------------------------------------------   
%%--------------------------------------------------------------------------------   
%\section{Other Quantum Algorithms}
%
%\mmm{can Shor's algorithm be treated dynamically?}
%
