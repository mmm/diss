
\chapter{A Dynamical Approach to Quantum Algorithms}
\label{chap:dynamics}
\index{Quantum Algorithms@\emph{Quantum Algorithms}!Dynamical Approach@\emph{Dynamical Approach}}%

%--------------------------------------------------------------------------------   
%--------------------------------------------------------------------------------   
\section{Grover}

%--------------------------------------------------------------------------------   
%\subsection{Sines and Cosines}
\subsection{Grover is Optimal}
\label{ssec:sincos}

Grover's algorithm was introduced and discussed at great length in
chapter (\ref{chap:quantumAlgorithms}).  There, it was mentioned that
Grover's algorithm can be seen as a rotation of states in Hilbert
space.  This section will establish this result a little more rigorously 
and show how it is used in an analysis of the complexity of the algorithm.

Consider the discrete action of Grover's algorithm as described in the end of
section (\ref{sec:grover}).  This will describe a search for \emph{one},
$\ket\omega$,
out of a list of $N$ states.  At any given step in the process, the
quantum register is in the state
\begin{equation}
\ket{\Psi} = k\ket{\omega} + \sum_{j\in\omega_\perp}l\ket{j},
\end{equation}
for some $k$ and $l$, which, at say the $i$-th iteration, 
look like
\begin{equation}
\begin{split}
k_{i+1} =& \frac{N-2}{N}k_i + \frac{2(N-1)}{N}l_i\\
l_{i+1} =& -\frac{2}{N}k_i + \frac{N-2}{N}l_i.
\end{split}
\label{e:discsincos}
\end{equation}

\begin{prop}
The coefficients $k$ and $l$ in (\ref{e:discsincos}) obey
\begin{equation}
\begin{split}
k_i =& \sin\left[\left(2i+1\right)\theta\right]\\
l_i =& \frac{1}{\sqrt{N-1}}\cos\left[\left(2i+1\right)\theta\right]
\end{split}
\label{e:sincos}
\end{equation}
for $\sin\theta = \frac{1}{\sqrt{N}}$.
\end{prop}

\begin{proof}
(By induction)  
Starting at step zero with an equally weighted 
superposition of all states
\begin{equation}
\ket{\Psi_0} = \frac{1}{\sqrt{N}}\ket{\omega} 
                + \sum_{j\in\omega_\perp}\frac{1}{\sqrt{N}}\ket{j}
\end{equation}
implies $k_0 = l_0 = \frac{1}{\sqrt{N}}$.
Inserting $i=0$ into (\ref{e:sincos}), you get
\begin{equation}
k_0 = \sin\left(\theta\right) = \frac{1}{\sqrt{N}}
\end{equation}
and 
\begin{equation}
\begin{split}
l_0 =& \frac{1}{\sqrt{N-1}}\cos\left(\theta\right)\\
    =& \frac{1}{\sqrt{N-1}}\sqrt{ 1 - \sin^2\left(\theta\right) }\\
    =& \frac{1}{\sqrt{N-1}}\sqrt{ 1 - \frac{1}{N} }
    = \frac{1}{\sqrt{N}}.
\end{split}
\end{equation}
So, the proposition holds for $i=0$.

Now, consider the $(i+1)$--th iteration
\begin{equation}
\begin{split}
&\begin{cases}
k_{i+1} =& \sin\left[\left(2(i+1)+1\right)\theta\right]\\
l_{i+1} =& \frac{1}{\sqrt{N-1}}\cos\left[\left(2(i+1)+1\right)\theta\right]
\end{cases}\\
&\begin{cases}
k_{i+1} =& \sin\left[\left(2i+1\right)\theta+2\theta\right]\\
l_{i+1} =& \frac{1}{\sqrt{N-1}}\cos\left[\left(2i+1\right)+2\theta\right]
\end{cases}\\
&\begin{cases}
k_{i+1} =& \sin\left[\left(2i+1\right)\theta\right]
           \cos\left(2\theta\right)
          +\cos\left[\left(2i+1\right)\theta\right]
           \sin\left(2\theta\right) \\
l_{i+1} =& \frac{1}{\sqrt{N-1}}\left\lbrace
                \cos\left[\left(2i+1\right)\theta\right]
                \cos\left(2\theta\right)
               -\sin\left[\left(2i+1\right)\theta\right]
                \sin\left(2\theta\right) 
           \right\rbrace.
\end{cases}
\end{split}
\end{equation}
By the induction hypothesis, 
\begin{equation}
\begin{cases}
k_{i+1} =& k_i\cos\left(2\theta\right)
          +\sqrt{N-1}\;l_i\sin\left(2\theta\right) \\
l_{i+1} =& \frac{1}{\sqrt{N-1}}\left\lbrace
                \sqrt{N-1}\;l_i\cos\left(2\theta\right)
               - k_i\sin\left(2\theta\right) 
           \right\rbrace,
\end{cases}
\end{equation}
and since
\begin{equation}
\sin\left(\theta\right) = \frac{1}{\sqrt{N}}
\quad\Rightarrow\quad
\cos\left(2\theta\right) = \frac{N-2}{N}
\quad\text{and}\quad
\sin\left(2\theta\right) = \frac{2\sqrt{N-1}}{N},
\end{equation}
then
\begin{equation}
\begin{cases}
k_{i+1} =& k_i\frac{N-2}{N} + l_i\frac{2(N-1)}{N}\\
l_{i+1} =& l_i\frac{N-2}{N} -k_i\frac{2}{N}. 
\end{cases}
\end{equation}
Since this is (\ref{e:discsincos}) as expected, $j\Rightarrow j+1$.
\end{proof}

\mmm{add Zalka stuff here}
\mmm{add in multisearch stopping time here too (from Boyer et al)}

%--------------------------------------------------------------------------------   
\subsection{Grover is a Geodesic}

Section (\ref{ssec:sincos}) expressed Grover's algorithm in terms of
discrete recurrence relations.  In the algorithm, the coefficients of
the state
\begin{equation}
\ket{\Psi} = k\ket{\omega} + \sum_{j\in\omega_\perp}l\ket{j}
\end{equation}
evolve according to
\begin{equation}
\begin{split}
k_i =& \sin\left[\left(2i+1\right)\theta\right]\\
l_i =& \frac{1}{\sqrt{N-1}}\cos\left[\left(2i+1\right)\theta\right].
\end{split}
\end{equation}
This evolution can be viewed as a discrete path in state space
as shown in figure (\ref{fig:discreteGrover}).
\begin{figure}[h]
\begin{center}
\begin{picture}(200,150)
    \thicklines
    \qbezier(0,70)(40,150)(120,100)
    \qbezier(120,100)(150,80)(180,85)
    \qbezier(0,70)(-40,-50)(140,0)
    \qbezier(180,85)(270,90)(200,30)
    \qbezier(200,30)(170,10)(140,0)
    \put(20,20){$\ket{\Psi}$}
    \put(110,70){$\ket{\omega}$}
    \qbezier[15](25,35)(70,70)(175,60)
\end{picture}
\caption{Grover's algorithm is a discrete path in state space 
starting from an equally--weighted superposition $\ket\Psi$ and 
continuing on through the desired state $\ket\omega$.  Note that
the algorithm keeps going and must be stopped at the anticipated
running time (When the algorithm will be near the desired state).}
\label{fig:discreteGrover}
\end{center}
\end{figure}

Now, consider the continuous path on the space of states
given by
\begin{equation}
\gamma\colon \R\to\CP^N\colon t\mapsto 
\begin{pmatrix}
    \sin(t)\\
    \cos(t)\\
    \vdots\\
    \cos(t)
\end{pmatrix}.
\label{e:gamma}
\end{equation}
The discrete points generated by Grover's algorithm 
{\sl lie upon this path}.  This path can then be used
as a {\sl continuous} approximation to Grover's algorithm
(figure \ref{fig:contGrover}).
\begin{figure}[h]
\begin{center}
\begin{picture}(200,150)
    \thicklines
    \qbezier(0,70)(40,150)(120,100)
    \qbezier(120,100)(150,80)(180,85)
    \qbezier(0,70)(-40,-50)(140,0)
    \qbezier(180,85)(270,90)(200,30)
    \qbezier(200,30)(170,10)(140,0)
    \put(20,20){$\ket{\Psi}$}
    \put(110,70){$\ket{\omega}$}
    \qbezier(25,35)(70,70)(175,60)
\end{picture}
\caption{Grover's algorithm can be approximated by a continuous
path in state space.  Note that this path coincides with the large
database limit of the algorithm and is a geodesic of the Fubini--Study
metric.}
\label{fig:contGrover}
\end{center}
\end{figure}

\begin{prop}
The path $\gamma$, as defined in (\ref{e:gamma}),
is a geodesic of the Fubini--Study metric on $\CP^N$.
\end{prop}

\begin{proof}
In homogeneous coordinates, $\gamma$ is easily seen to be
a geodesic of the $(2N+1)$--sphere.
Since $\CP^N$ can be written as 
\begin{equation}
S^1\to S^{2N+1}\to \CP^N,
\end{equation}
then the geodesic projects down onto a geodesic of the base.
\mmm{grin}
\end{proof}


\begin{prop}
$\gamma$ coincides with the large database
limit of Grover's algorithm.
\end{prop}

\begin{proof}
Duh!
\end{proof}

%--------------------------------------------------------------------------------   
%--------------------------------------------------------------------------------   
\section{Is there a Hamiltonian?}

\mmm{Used to be Trotter Formula}

The diffusion transformation in Grover's algorithm,
as defined in section (\ref{ssec:diffusion}), looks like
\begin{equation}
\begin{pmatrix}
    \frac{2-N}{N} & \frac{2}{N} & \cdots & \frac{2}{N} \\
    \frac{2}{N}   & \frac{2-N}{N} & &  \\
    \vdots &  & \ddots & \vdots \\
    \frac{2}{N}   & \cdots && \frac{2-N}{N} 
\end{pmatrix}.
\end{equation}
This can be rewritten as
\begin{equation}
\frac{2}{N}
\begin{pmatrix}
    1 & \cdots & 1 \\
    \vdots & \ddots & \vdots \\
    1 & \cdots & 1
\end{pmatrix}
- \textbf{1}.
\end{equation}


%--------------------------------------------------------------------------------   
%--------------------------------------------------------------------------------   
\section{Dynamical Maps}

\subsubsection{Intro to dynamical maps}
Open systems.
What the hell are dynamical maps?

Unfortunatley, physical systems do not exist in isolation.  There
are always interactions with environments or baths that cause
quantum mechanical systems to evolve by possibly more complicated
means than unitary transformations.

A somewhat more general approach to quantum evolution is to 
consider dynamical evolution of a quantum mechanical system
described as simply a map of density matrices to density
matrices, the so--called \emph{dynamical map}.

An even more general approach is to simply look at maps of
density matrices, where the image itself doesn't necessarily have to
be a physical density matrix.  It could be simply a subsystem
of a physical system, and hence it's trace could be less than
one.

\mmm{Phase space stuff or configuration space stuff}

%--------------------------------------------------------------------------------   
\subsection{Numerical Model of Grover's Algorithm}

\subsubsection{The actual model}

\subsubsection{Simulation of Grover is a dynamical map}

\subsubsection{How this relates to Noise}


\mmm{show some baseline results before noise or stability is even considered???}
\mmm{low dimensional examples of flow to do analytically and compare}


%--------------------------------------------------------------------------------   
%--------------------------------------------------------------------------------   
\section{Other Quantum Algorithms}

Shor?
