
\chapter{Conclusions}
\label{chap:conclusion}
\index{Conclusion@\emph{Conclusion}}%


In discussing results, the first thing to note is that
a quantum computer sure would have been useful in actually
performing this simulation.
As Feynman noted\cite{Feynman:82}, simulating a quantum system
by a classical computer requires an exponentially large amount
of overhead.  This was clearly evident in this research.

Decoherence and the 
experimental control and manipulation of different 
physical devices introduce noise into a computation.  
How much noise can the computation withstand before
it fails?  Exactly how does a computation fail in the
presence of various types of noise?
Any means of getting a handle
on the stability of these algorithms warrants attention.

The current work uses a numerical simulation of
a dynamical description of Grover's quantum search
algorithm to determine the answers to some of these
questions.

\pagebreak
%--------------------------------------------------------------------------------   
%--------------------------------------------------------------------------------   
\section{The Effects of Noise on Grover's Algorithm}

Noise does not slow the algorithm down.  It only reduces the
probability that the algorithm will succeed.  This is shown in 
figure \ref{fig:humps}.
\begin{figure}[h]
\begin{center}
\epsfig{file=figures/humps.pstex,height=3.5in,width=4.5in}
\end{center}
\caption{Probability of success -v- time for different noise levels and different numbers
of qubits.  Black is a run without noise, red is the same run with noise.  Notice that
where the peaks occur in ``time'' do not change due to noise.  The relative times of the
peaks are artificial in this graph.}
\label{fig:humps}
\end{figure}

\pagebreak

\noindent
Define the maximum amount of noise that the algorithm can tolerate
as the point at which the probability of success falls below $\frac{1}{2}$.
For $n$ qubits,
this maximum amount of tolerable noise falls off as $\frac{1}{n^3}$, 
as shown in figure \ref{fig:results}.
\begin{figure}[h]
\begin{center}
\epsfig{file=figures/max-noise.pstex,height=3.5in,width=4.5in}
\end{center}
\caption{Maximum tolerable noise as a function of the number of qubits.
This curve is approximately fit by $y = 0.5812 x^{-2.886}$.}
\label{fig:results}
\end{figure}
%y = 0.58123479 N^{-2.886221}

\pagebreak

\noindent
Appendix \ref{chap:data} contains:
\begin{itemize}
\item Bures-v-time for different noise levels and different numbers of qubits.
%\item Probability of success -v- time for different noise levels and different numbers
%of qubits.
\item maximum probability of success -v- noise level for given number of qubits.
\item maximum probability of success -v- number of qubits for given noise levels.
\end{itemize}

%Want the max allowed noise so that maximum target coefficient is still above 
%$\frac{1}{2}$.  Really want max allowed noise so that the algorithm is still
%better (faster?) than the classical version.  How does this whole
%$P_{\text{cutoff}}$ thing work?
%\cite{Pablo-Norman/Ruiz-Altaba:99}


%%--------------------------------------------------------------------------------   
%\subsection{Physical Implications}
%
%\begin{itemize}
%\item Give an example where the physical noise ends up as noise
%in the state vector... should be easy.
%\item How does noise in a gate propagate out?
%\end{itemize}


%%--------------------------------------------------------------------------------   
%%--------------------------------------------------------------------------------   
%\section{Quantum Error Correction}
%
%\begin{itemize}
%\item ECCs
%\item How does this compare with bit--flips?
%\end{itemize}

%--------------------------------------------------------------------------------   
%--------------------------------------------------------------------------------   
\section{Topics for Further Investigation}

\begin{enumerate}
\item How much does Quantum error correction help?
\begin{itemize}
\item ECCs
\item How does this compare with bit--flips?
\end{itemize}
\item Decoherence--free subspaces?
\item Shor dynamically?
\end{enumerate}
