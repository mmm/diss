
\chapter{Conclusions}
\label{chap:conclusion}
\index{Conclusion@\emph{Conclusion}}%

%--------------------------------------------------------------------------------   
%--------------------------------------------------------------------------------   
\section{The Effects of Noise on Grover's Algorithm}

\subsubsection{Noise does not slow the algorithm down}

\begin{figure}[h]
\begin{center}
\epsfig{file=figures/max-noise.pstex,height=3.5in,width=4.5in}
\end{center}
\caption{Maximum tolerable noise as a function of the number of qubits.}
\end{figure}

\begin{equation}
y = 0.58123479 N^{-2.886221}
\end{equation}


(show pictures with humps)

Will have:
\begin{itemize}
\item Bures-v-time for different noise levels and different numbers of qubits.
\item target coefficients -v- time for different noise levels and different numbers
of qubits.
\item maximum target coefficient -v- noise level for given number of qubits.
\item maximum target coefficient -v- number of qubits for given noise levels.
\end{itemize}
Want the max allowed noise so that maximum target coefficient is still above 
$\frac{1}{2}$.  Really want max allowed noise so that the algorithm is still
better (faster?) than the classical version.  How does this whole
$P_{\text{cutoff}}$ thing work?\cite{Pablo-Norman/Ruiz-Altaba:99}

Show how the time to finish remains the same\dots How?



%--------------------------------------------------------------------------------   
\subsection{Physical Implications}

\begin{itemize}
\item How does noise in gates propagate out?
\item Can anything be said about chaos?
\item How does this compare with bit--flips?
\end{itemize}



%--------------------------------------------------------------------------------   
%--------------------------------------------------------------------------------   
\section{Quantum Error Correction}

\subsubsection{ECCs}

%--------------------------------------------------------------------------------   
%--------------------------------------------------------------------------------   
\section{Topics for Further Investigation}

\begin{enumerate}
\item How much do QECC's help?
\item Decoherence free subspaces?
\item Shor?
\end{enumerate}
