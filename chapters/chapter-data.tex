\chapter{The Data}
\label{chap:data}
\index{Data@\emph{Data}}%

\newlength{\dataheight}
\setlength{\dataheight}{4in}
\newlength{\datawidth}
\setlength{\datawidth}{4.5in}

Here, the data is presented as a series of figures organized
into sections depending upon the information displayed.

Uncertainties in the numerical results are due almost exclusively
to roundoff error.  The computer is limited in the amount of space
used to approximate a real number, and this limitation causes rounding
to occur at every step of the calculation.  These uncertainties
are difficult to explicitly propagate through the calculations, but
a good approximation of roundoff errors can be obtained by adjusting
the resolution of integration.  The limits of uncertainty in the
following figures were obtained by doubling/halving the step size
while simultaneously halving/doubling the number of steps taken.

\pagebreak

%----------------------------------------------------------------------
%----------------------------------------------------------------------
\section{Probability of success -v- maximum noise level for various
numbers of qubits}


%----------------------------------------------------------------------
% leading coeff -v- noise for fixed number of qubits

%\input{figures/targetCoeff-max-q3.pstex_t}
\begin{figure}[h]
\begin{center}
\epsfig{file=figures/targetCoeff-max-q3.pstex,height=\dataheight,width=\datawidth}
\end{center}
\caption{Probability of sucess plotted as a function of the maximum noise allowed.
This is done for 3 qubits.}
\end{figure}

\pagebreak

\begin{figure}[h]
\begin{center}
\epsfig{file=figures/targetCoeff-max-q4.pstex,height=\dataheight,width=\datawidth}
\end{center}
\caption{Probability of sucess plotted as a function of the maximum noise allowed.
This is done for 4 qubits.}
\end{figure}

\pagebreak

\begin{figure}[h]
\begin{center}
\epsfig{file=figures/targetCoeff-max-q5.pstex,height=\dataheight,width=\datawidth}
\end{center}
\caption{Probability of sucess plotted as a function of the maximum noise allowed.
This is done for 5 qubits.}
\end{figure}

\pagebreak

\begin{figure}[h]
\begin{center}
\epsfig{file=figures/targetCoeff-max-q6.pstex,height=\dataheight,width=\datawidth}
\end{center}
\caption{Probability of sucess plotted as a function of the maximum noise allowed.
This is done for 6 qubits.}
\end{figure}

\pagebreak

\begin{figure}[h]
\begin{center}
\epsfig{file=figures/targetCoeff-max-q7.pstex,height=\dataheight,width=\datawidth}
\end{center}
\caption{Probability of sucess plotted as a function of the maximum noise allowed.
This is done for 7 qubits.}
\end{figure}

\pagebreak

%----------------------------------------------------------------------
%----------------------------------------------------------------------
\vfill
\section{Probability of success -v- number of qubits for various amounts of noise}
\vfill

\pagebreak

%----------------------------------------------------------------------
% leading coeff -v- qubit for fixed noise

\begin{figure}[h]
\begin{center}
\epsfig{file=figures/targetCoeff-max-n0.001.pstex,height=\dataheight,width=\datawidth}
\end{center}
\caption{Probability of sucess plotted as a function of the number of qubits. This
is done for a noise threshold of 0.001.}
\end{figure}

\pagebreak

\begin{figure}[h]
\begin{center}
\epsfig{file=figures/targetCoeff-max-n0.0025.pstex,height=\dataheight,width=\datawidth}
\end{center}
\caption{Probability of sucess plotted as a function of the number of qubits. This
is done for a noise threshold of 0.0025.}
\end{figure}

\pagebreak

\begin{figure}[h]
\begin{center}
\epsfig{file=figures/targetCoeff-max-n0.005.pstex,height=\dataheight,width=\datawidth}
\end{center}
\caption{Probability of sucess plotted as a function of the number of qubits. This
is done for a noise threshold of 0.005.}
\end{figure}

\pagebreak

\begin{figure}[h]
\begin{center}
\epsfig{file=figures/targetCoeff-max-n0.0075.pstex,height=\dataheight,width=\datawidth}
\end{center}
\caption{Probability of sucess plotted as a function of the number of qubits. This
is done for a noise threshold of 0.0075.}
\end{figure}


\pagebreak

%----------------------------------------------------------------------
%----------------------------------------------------------------------
\vfill
\section{Bures distance -v- pseudotime for various numbers of qubits and
amounts of noise}
\label{sec:BuresData}
\vfill

\pagebreak

%----------------------------------------------------------------------
% Bures -v- pseudotime for fixed noise level

%
%\begin{figure}[h]
%\begin{center}
%\epsfig{file=figures/Bures-2q-0.0001.pstex,height=\dataheight,width=\datawidth}
%\end{center}
%\caption{Bures distance as a function of time for 2 qubits with noise
%threshold of 0.0001. }
%\end{figure}
%
%\pagebreak
%
%\begin{figure}[h]
%\begin{center}
%\epsfig{file=figures/Bures-2q-0.001.pstex,height=\dataheight,width=\datawidth}
%\end{center}
%\caption{Bures distance as a function of time for 2 qubits with noise
%threshold of 0.001. }
%\end{figure}
%
%\pagebreak
%
%\begin{figure}[h]
%\begin{center}
%\epsfig{file=figures/Bures-2q-0.01.pstex,height=\dataheight,width=\datawidth}
%\end{center}
%\caption{Bures distance as a function of time for 2 qubits with noise
%threshold of 0.01. }
%\end{figure}
%
%\pagebreak

\begin{figure}[h]
\begin{center}
\epsfig{file=figures/Bures-3q-0.0001.pstex,height=\dataheight,width=\datawidth}
\end{center}
\caption{Bures distance as a function of time for 3 qubits with noise
threshold of 0.0001. }
\end{figure}

\pagebreak

\begin{figure}[h]
\begin{center}
\epsfig{file=figures/Bures-3q-0.001.pstex,height=\dataheight,width=\datawidth}
\end{center}
\caption{Bures distance as a function of time for 3 qubits with noise
threshold of 0.001. }
\end{figure}

\pagebreak

\begin{figure}[h]
\begin{center}
\epsfig{file=figures/Bures-3q-0.01.pstex,height=\dataheight,width=\datawidth}
\end{center}
\caption{Bures distance as a function of time for 3 qubits with noise
threshold of 0.01. }
\end{figure}

\pagebreak

\begin{figure}[h]
\begin{center}
\epsfig{file=figures/Bures-4q-0.0001.pstex,height=\dataheight,width=\datawidth}
\end{center}
\caption{Bures distance as a function of time for 4 qubits with noise
threshold of 0.0001. }
\end{figure}

\pagebreak

\begin{figure}[h]
\begin{center}
\epsfig{file=figures/Bures-4q-0.001.pstex,height=\dataheight,width=\datawidth}
\end{center}
\caption{Bures distance as a function of time for 4 qubits with noise
threshold of 0.001. }
\end{figure}

\pagebreak

\begin{figure}[h]
\begin{center}
\epsfig{file=figures/Bures-4q-0.01.pstex,height=\dataheight,width=\datawidth}
\end{center}
\caption{Bures distance as a function of time for 4 qubits with noise
threshold of 0.01. }
\end{figure}

\pagebreak

\begin{figure}[h]
\begin{center}
\epsfig{file=figures/Bures-5q-0.0001.pstex,height=\dataheight,width=\datawidth}
\end{center}
\caption{Bures distance as a function of time for 5 qubits with noise
threshold of 0.0001. }
\end{figure}

\pagebreak

\begin{figure}[h]
\begin{center}
\epsfig{file=figures/Bures-5q-0.001.pstex,height=\dataheight,width=\datawidth}
\end{center}
\caption{Bures distance as a function of time for 5 qubits with noise
threshold of 0.001.}
\end{figure}

\pagebreak

\begin{figure}[h]
\begin{center}
\epsfig{file=figures/Bures-5q-0.01.pstex,height=\dataheight,width=\datawidth}
\end{center}
\caption{Bures distance as a function of time for 5 qubits with noise
threshold of 0.01.}
\end{figure}

