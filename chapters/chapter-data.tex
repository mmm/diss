\chapter{Data}
\label{chap:data}
\index{Data@\emph{Data}}%

\newlength{\dataheight}
\setlength{\dataheight}{3in}
\newlength{\datawidth}
\setlength{\datawidth}{4.5in}

%----------------------------------------------------------------------
%----------------------------------------------------------------------
\section{Probability of success -v- maximum tolerable noise for various
numbers of qubits}

\pagebreak

%----------------------------------------------------------------------
% leading coeff -v- noise for fixed number of qubits

\begin{figure}[h]
\begin{center}
\epsfig{file=figures/targetCoeff-max-q3.pstex,height=\dataheight,width=\datawidth}
\end{center}
\caption{Probability of sucess plotted as a function of the maximum noise allowed.
This is done for 3 qubits.}
\end{figure}

\pagebreak

\begin{figure}[h]
\begin{center}
\epsfig{file=figures/targetCoeff-max-q4.pstex,height=\dataheight,width=\datawidth}
\end{center}
\caption{Probability of sucess plotted as a function of the maximum noise allowed.
This is done for 4 qubits.}
\end{figure}

\pagebreak

\begin{figure}[h]
\begin{center}
\epsfig{file=figures/targetCoeff-max-q5.pstex,height=\dataheight,width=\datawidth}
\end{center}
\caption{Probability of sucess plotted as a function of the maximum noise allowed.
This is done for 5 qubits.}
\end{figure}

\pagebreak

\begin{figure}[h]
\begin{center}
\epsfig{file=figures/targetCoeff-max-q6.pstex,height=\dataheight,width=\datawidth}
\end{center}
\caption{Probability of sucess plotted as a function of the maximum noise allowed.
This is done for 6 qubits.}
\end{figure}

\pagebreak

\begin{figure}[h]
\begin{center}
\epsfig{file=figures/targetCoeff-max-q7.pstex,height=\dataheight,width=\datawidth}
\end{center}
\caption{Probability of sucess plotted as a function of the maximum noise allowed.
This is done for 7 qubits.}
\end{figure}

\pagebreak

%----------------------------------------------------------------------
%----------------------------------------------------------------------
\vfill
\section{Probability of success -v- number of qubits for various amounts of noise}
\vfill

\pagebreak

%----------------------------------------------------------------------
% leading coeff -v- qubit for fixed noise

\begin{figure}[h]
\begin{center}
\epsfig{file=figures/targetCoeff-max-n0.001.pstex,height=\dataheight,width=\datawidth}
\end{center}
\caption{Probability of sucess plotted as a function of the number of qubits. This
is done for a noise threshold of 0.001.}
\end{figure}

\pagebreak

\begin{figure}[h]
\begin{center}
\epsfig{file=figures/targetCoeff-max-n0.0025.pstex,height=\dataheight,width=\datawidth}
\end{center}
\caption{Probability of sucess plotted as a function of the number of qubits. This
is done for a noise threshold of 0.0025.}
\end{figure}

\pagebreak

\begin{figure}[h]
\begin{center}
\epsfig{file=figures/targetCoeff-max-n0.005.pstex,height=\dataheight,width=\datawidth}
\end{center}
\caption{Probability of sucess plotted as a function of the number of qubits. This
is done for a noise threshold of 0.005.}
\end{figure}

\pagebreak

\begin{figure}[h]
\begin{center}
\epsfig{file=figures/targetCoeff-max-n0.0075.pstex,height=\dataheight,width=\datawidth}
\end{center}
\caption{Probability of sucess plotted as a function of the number of qubits. This
is done for a noise threshold of 0.0075.}
\end{figure}


\pagebreak

%----------------------------------------------------------------------
%----------------------------------------------------------------------
\vfill
\section{Bures distance -v- pseudotime for various numbers of qubits and
amounts of noise}
\vfill

\pagebreak

%----------------------------------------------------------------------
% Bures -v- pseudotime for fixed noise level


\begin{figure}[h]
\begin{center}
\epsfig{file=figures/Bures-2q-0.0001.pstex,height=\dataheight,width=\datawidth}
\end{center}
\caption{Bures distance as a function of time for 2 qubits with noise
threshold of 0.0001.  Uncertainties are included and are characterized 
by changing the resolution of the numerical integration.}
\end{figure}

\pagebreak

\begin{figure}[h]
\begin{center}
\epsfig{file=figures/Bures-2q-0.001.pstex,height=\dataheight,width=\datawidth}
\end{center}
\caption{Bures distance as a function of time for 2 qubits with noise
threshold of 0.001.  Uncertainties are included and are characterized 
by changing the resolution of the numerical integration.}
\end{figure}

\pagebreak

\begin{figure}[h]
\begin{center}
\epsfig{file=figures/Bures-2q-0.01.pstex,height=\dataheight,width=\datawidth}
\end{center}
\caption{Bures distance as a function of time for 2 qubits with noise
threshold of 0.01.  Uncertainties are included and are characterized 
by changing the resolution of the numerical integration.}
\end{figure}

\pagebreak

\begin{figure}[h]
\begin{center}
\epsfig{file=figures/Bures-3q-0.0001.pstex,height=\dataheight,width=\datawidth}
\end{center}
\caption{Bures distance as a function of time for 3 qubits with noise
threshold of 0.0001.  Uncertainties are included and are characterized 
by changing the resolution of the numerical integration.}
\end{figure}

\pagebreak

\begin{figure}[h]
\begin{center}
\epsfig{file=figures/Bures-3q-0.001.pstex,height=\dataheight,width=\datawidth}
\end{center}
\caption{Bures distance as a function of time for 3 qubits with noise
threshold of 0.001.  Uncertainties are included and are characterized 
by changing the resolution of the numerical integration.}
\end{figure}

\pagebreak

\begin{figure}[h]
\begin{center}
\epsfig{file=figures/Bures-3q-0.01.pstex,height=\dataheight,width=\datawidth}
\end{center}
\caption{Bures distance as a function of time for 3 qubits with noise
threshold of 0.01.  Uncertainties are included and are characterized 
by changing the resolution of the numerical integration.}
\end{figure}

\pagebreak

\begin{figure}[h]
\begin{center}
\epsfig{file=figures/Bures-4q-0.0001.pstex,height=\dataheight,width=\datawidth}
\end{center}
\caption{Bures distance as a function of time for 4 qubits with noise
threshold of 0.0001.  Uncertainties are included and are characterized 
by changing the resolution of the numerical integration.}
\end{figure}

\pagebreak

\begin{figure}[h]
\begin{center}
\epsfig{file=figures/Bures-4q-0.001.pstex,height=\dataheight,width=\datawidth}
\end{center}
\caption{Bures distance as a function of time for 4 qubits with noise
threshold of 0.001.  Uncertainties are included and are characterized 
by changing the resolution of the numerical integration.}
\end{figure}

\pagebreak

\begin{figure}[h]
\begin{center}
\epsfig{file=figures/Bures-4q-0.01.pstex,height=\dataheight,width=\datawidth}
\end{center}
\caption{Bures distance as a function of time for 4 qubits with noise
threshold of 0.01.  Uncertainties are included and are characterized 
by changing the resolution of the numerical integration.}
\end{figure}

\pagebreak

\begin{figure}[h]
\begin{center}
\epsfig{file=figures/Bures-5q-0.0001.pstex,height=\dataheight,width=\datawidth}
\end{center}
\caption{Bures distance as a function of time for 5 qubits with noise
threshold of 0.0001.  Uncertainties are included and are characterized 
by changing the resolution of the numerical integration.}
\end{figure}

\pagebreak

\begin{figure}[h]
\begin{center}
\epsfig{file=figures/Bures-5q-0.001.pstex,height=\dataheight,width=\datawidth}
\end{center}
\caption{Bures distance as a function of time for 5 qubits with noise
threshold of 0.001.  Uncertainties are included and are characterized 
by changing the resolution of the numerical integration.}
\end{figure}

\pagebreak

\begin{figure}[h]
\begin{center}
\epsfig{file=figures/Bures-5q-0.01.pstex,height=\dataheight,width=\datawidth}
\end{center}
\caption{Bures distance as a function of time for 5 qubits with noise
threshold of 0.01.  Uncertainties are included and are characterized 
by changing the resolution of the numerical integration.}
\end{figure}

