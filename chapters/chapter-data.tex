\chapter{Data}
\label{chap:data}
\index{Data@\emph{Data}}%





\begin{figure}
\begin{center}
\epsfig{file=figures/Bures-2q-0.0001.pstex,height=3.5in,width=4in}
\end{center}
\caption{Bures distance as a function of time for 2 qubits with noise
threshold of 0.0001.  Uncertainties are included and are characterized 
by changing the resolution of the numerical integration.}
\end{figure}

\pagebreak

\begin{figure}
\begin{center}
\epsfig{file=figures/Bures-2q-0.001.pstex,height=3.5in,width=4in}
\end{center}
\caption{Bures distance as a function of time for 2 qubits with noise
threshold of 0.001.  Uncertainties are included and are characterized 
by changing the resolution of the numerical integration.}
\end{figure}

\pagebreak

\begin{figure}
\begin{center}
\epsfig{file=figures/Bures-2q-0.01.pstex,height=3.5in,width=4in}
\end{center}
\caption{Bures distance as a function of time for 2 qubits with noise
threshold of 0.01.  Uncertainties are included and are characterized 
by changing the resolution of the numerical integration.}
\end{figure}

\pagebreak

\begin{figure}
\begin{center}
\epsfig{file=figures/Bures-3q-0.0001.pstex,height=3.5in,width=4in}
\end{center}
\caption{Bures distance as a function of time for 3 qubits with noise
threshold of 0.0001.  Uncertainties are included and are characterized 
by changing the resolution of the numerical integration.}
\end{figure}

\pagebreak

\begin{figure}
\begin{center}
\epsfig{file=figures/Bures-3q-0.001.pstex,height=3.5in,width=4in}
\end{center}
\caption{Bures distance as a function of time for 3 qubits with noise
threshold of 0.001.  Uncertainties are included and are characterized 
by changing the resolution of the numerical integration.}
\end{figure}

\pagebreak

\begin{figure}
\begin{center}
\epsfig{file=figures/Bures-3q-0.01.pstex,height=3.5in,width=4in}
\end{center}
\caption{Bures distance as a function of time for 3 qubits with noise
threshold of 0.01.  Uncertainties are included and are characterized 
by changing the resolution of the numerical integration.}
\end{figure}

\pagebreak

\begin{figure}
\begin{center}
\epsfig{file=figures/Bures-4q-0.0001.pstex,height=3.5in,width=4in}
\end{center}
\caption{Bures distance as a function of time for 4 qubits with noise
threshold of 0.0001.  Uncertainties are included and are characterized 
by changing the resolution of the numerical integration.}
\end{figure}

\pagebreak

\begin{figure}
\begin{center}
\epsfig{file=figures/Bures-4q-0.001.pstex,height=3.5in,width=4in}
\end{center}
\caption{Bures distance as a function of time for 4 qubits with noise
threshold of 0.001.  Uncertainties are included and are characterized 
by changing the resolution of the numerical integration.}
\end{figure}

\pagebreak

\begin{figure}
\begin{center}
\epsfig{file=figures/Bures-4q-0.01.pstex,height=3.5in,width=4in}
\end{center}
\caption{Bures distance as a function of time for 4 qubits with noise
threshold of 0.01.  Uncertainties are included and are characterized 
by changing the resolution of the numerical integration.}
\end{figure}

\pagebreak

\begin{figure}
\begin{center}
\epsfig{file=figures/Bures-5q-0.0001.pstex,height=3.5in,width=4in}
\end{center}
\caption{Bures distance as a function of time for 5 qubits with noise
threshold of 0.0001.  Uncertainties are included and are characterized 
by changing the resolution of the numerical integration.}
\end{figure}

\pagebreak

\begin{figure}
\begin{center}
\epsfig{file=figures/Bures-5q-0.001.pstex,height=3.5in,width=4in}
\end{center}
\caption{Bures distance as a function of time for 5 qubits with noise
threshold of 0.001.  Uncertainties are included and are characterized 
by changing the resolution of the numerical integration.}
\end{figure}

\pagebreak

\begin{figure}
\begin{center}
\epsfig{file=figures/Bures-5q-0.01.pstex,height=5in,width=5in,angle=90}
\end{center}
\caption{Bures distance as a function of time for 5 qubits with noise
threshold of 0.01.  Uncertainties are included and are characterized 
by changing the resolution of the numerical integration.}
\end{figure}

\pagebreak


%mmm
\begin{figure}
\begin{center}
\epsfig{file=figures/targetCoeff-max-n0.001.pstex,height=5in,width=5in,angle=90}
\end{center}
\caption{Probability of sucess plotted as a function of the number of qubits. This
is done for a noise threshold of 0.001.}
\end{figure}

\pagebreak

\begin{figure}
\begin{center}
\epsfig{file=figures/targetCoeff-max-n0.0025.pstex,height=5in,width=5in,angle=90}
\end{center}
\caption{Probability of sucess plotted as a function of the number of qubits. This
is done for a noise threshold of 0.0025.}
\end{figure}

\pagebreak

\begin{figure}
\begin{center}
\epsfig{file=figures/targetCoeff-max-n0.005.pstex,height=5in,width=5in,angle=90}
\end{center}
\caption{Probability of sucess plotted as a function of the number of qubits. This
is done for a noise threshold of 0.005.}
\end{figure}

\pagebreak

\begin{figure}
\begin{center}
\epsfig{file=figures/targetCoeff-max-n0.0075.pstex,height=5in,width=5in,angle=90}
\end{center}
\caption{Probability of sucess plotted as a function of the number of qubits. This
is done for a noise threshold of 0.0075.}
\end{figure}

\pagebreak

%mmm
\begin{figure}
\begin{center}
\epsfig{file=figures/targetCoeff-max-q3.pstex,height=5in,width=5in,angle=90}
\end{center}
\caption{Probability of sucess plotted as a function of the maximum noise allowed.
This is done for 3 qubits.}
\end{figure}

\pagebreak

\begin{figure}
\begin{center}
\epsfig{file=figures/targetCoeff-max-q4.pstex,height=5in,width=5in,angle=90}
\end{center}
\caption{Probability of sucess plotted as a function of the maximum noise allowed.
This is done for 4 qubits.}
\end{figure}

\pagebreak

\begin{figure}
\begin{center}
\epsfig{file=figures/targetCoeff-max-q5.pstex,height=5in,width=5in,angle=90}
\end{center}
\caption{Probability of sucess plotted as a function of the maximum noise allowed.
This is done for 5 qubits.}
\end{figure}

%\pagebreak
%
%\begin{figure}
%\begin{center}
%\epsfig{file=figures/targetCoeff-max-q6.pstex,height=5in,width=5in,angle=90}
%\end{center}
%\caption{Probability of sucess plotted as a function of the maximum noise allowed.
%This is done for 6 qubits.}
%\end{figure}
%
%\pagebreak
%
%\begin{figure}
%\begin{center}
%\epsfig{file=figures/targetCoeff-max-q7.pstex,height=5in,width=5in,angle=90}
%\end{center}
%\caption{Probability of sucess plotted as a function of the maximum noise allowed.
%This is done for 7 qubits.}
%\end{figure}
