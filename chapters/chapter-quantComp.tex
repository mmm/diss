
\chapter{Quantum Computation}
\label{chap:quantumComputation}
\index{Quantum Computation@\emph{Quantum Computation}}%

\subsubsection{General QC intro}

%--------------------------------------------------------------------------------   
%--------------------------------------------------------------------------------   
\section{Information Theory}

\subsubsection{Why do we care?}

%--------------------------------------------------------------------------------   
\subsection{Classical}

\subsubsection{Entropy}
\subsubsection{Information}
\subsubsection{Coding}
\subsubsection{Channel Capacity}

%--------------------------------------------------------------------------------   
\subsection{Quantum}

\subsubsection{Entropy}
\subsubsection{Information}
\subsubsection{Measurement (POVMs and Optimal Measurements?)}
\subsubsection{Entanglement}
\subsubsection{Coding}
\subsubsection{Dense Coding and Channel Capacity}
\subsubsection{Teleportation}
\subsubsection{Cloning}

%--------------------------------------------------------------------------------   
%--------------------------------------------------------------------------------   
\section{The Theory of Computation}

\mmm{Show circuits for some explicit computable function}

%--------------------------------------------------------------------------------   
\subsection{Classical}

\subsubsection{Universal Computation}
\subsubsection{Classical Circuits}
\subsubsection{Complexity Classes}

%--------------------------------------------------------------------------------   
\subsection{Quantum}

\subsubsection{Reversibility}
\subsubsection{Universal Computation}
\subsubsection{Quantum Circuits}

\mmm{testing}
\begin{figure}
\begin{center}
\begin{picture}(100,50)
    \put(10,0){\framebox(20,20){\bf{$H$}}}
    \put(10,30){\framebox(20,20){\bf{$H$}}}
    \put(70,30){\framebox(20,20){\bf{$H$}}}
    \put(40,0){\framebox(20,20){\bf{$U_f$}}}
    \put(-20,7){$\ket{x}$}
    \put(-20,37){$\ket{0}$}
    \put(110,7){$\ket{f(x)}$}
    \put(110,37){$\ket{0}$}
    \put(50,40){\circle*{4}}
    \path(0,10)(10,10)
    \path(30,10)(40,10)
    \path(60,10)(100,10)
    \path(0,40)(10,40)
    \path(30,40)(70,40)
    \path(90,40)(100,40)
    \path(50,20)(50,40)
\end{picture}
\caption{Deutsch's circuit}
\end{center}
\end{figure}

\begin{figure}
\begin{center}
\begin{picture}(50,60)
    \put(20,0){\framebox(20,20){\bf{$H$}}}
    \put(-20,7){$\ket{1}$}
    \put(-20,37){$\ket{0}$}
    \put(30,40){\circle*{4}}
    \path(0,10)(20,10)
    \path(40,10)(60,10)
    \path(0,40)(60,40)
    \path(30,20)(30,40)
\end{picture}
\caption{A quantum circuit applying a Hadamaard transformation to one of
two qubits with intersecting lines}
\end{center}
\end{figure}

\begin{figure}
\begin{center}
\begin{picture}(40,50)
    \put(20,40){\circle*{4}}
    \put(20,10){\circle{6}}
    \put(-20,7){$\ket{y}$}
    \put(-20,37){$\ket{x}$}
    \put(50,7){$\ket{y\oplus x}$}
    \put(50,37){$\ket{x}$}
    \path(0,10)(40,10)
    \path(0,40)(40,40)
    \path(20,7)(20,40)
\end{picture}
\caption{A controlled--\texttt{NOT} (\texttt{CNOT}) gate}
\end{center}
\end{figure}

\begin{figure}
\begin{center}
\begin{picture}(40,80)
    \put(20,70){\circle*{4}}
    \put(20,40){\circle*{4}}
    \put(20,10){\circle{6}}
    \put(-20,7){$\ket{z}$}
    \put(-20,37){$\ket{y}$}
    \put(-20,67){$\ket{x}$}
    \put(50,7){$\ket{z\oplus y\oplus x}$}
    \put(50,37){$\ket{y}$}
    \put(50,67){$\ket{x}$}
    \path(0,10)(40,10)
    \path(0,40)(40,40)
    \path(0,70)(40,70)
    \path(20,7)(20,70)
\end{picture}
\caption{A controlled--controlled--\texttt{NOT} (\texttt{CCNOT}) gate}
\end{center}
\end{figure}





\subsubsection{Complexity Classes}
\subsubsection{The Hadamaard Transformation}
\index{Hadamaard transformation}

The one--dimensional Hadamaard transformation takes a quantum bit into the
superposition given by
\begin{equation}
\begin{split}
\ket{0}\quad\mapsto&\quad\frac{1}{\sqrt{2}}\left(\ket{0} +\ket{1}\right)\\
\ket{1}\quad\mapsto&\quad\frac{1}{\sqrt{2}}\left(\ket{0} -\ket{1}\right).
\end{split}
\end{equation}
This transformation has a matrix realization given by
\begin{equation}
H_1 
%\left[\begin{matrix}\ket{0}\\ \ket{1}\end{matrix}\right]
= \frac{1}{\sqrt{2}}
\left[\begin{matrix}1&1\\ 1&-1\end{matrix}\right].
%\left[\begin{matrix}\ket{0}\\ \ket{1}\end{matrix}\right].
\end{equation}

The version of the Hadamaard transformation that sees the most use is
the $N$--dimensional Hadamaard transformation defined as the $N$--fold
tensor product of one--dimensional Hadamaard transformations
\begin{equation}
H= H_N= \bigotimes^N H_1.
\end{equation}
This is often used as the first step in many quantum algorithms.
First, start with the \emph{zero register}
\index{zero register}
\begin{equation}
\ket{\bf 0}=\ket{0\ldots 0} = \bigotimes^N\ket{0},
\end{equation} and then act
\begin{equation}
\begin{split}
H\ket{\bf 0} =& H_N\ket{0\ldots 0}\\
=& \bigotimes^N H_1\ket{0}\\
=& \bigotimes^N \frac{1}{\sqrt{2}}\left(\ket{0} +\ket{1}\right)\\
=& \frac{1}{\sqrt{2^N}}\biggl\lbrace \ket{0\ldots 0} + \ket{0\ldots 01} +
\quad\cdots\quad + \ket{1\ldots 1}\biggr\rbrace.
\end{split}
\label{e:hadamaard}
\end{equation}
Now, each of the kets on the bottom--RHS of equation (\ref{e:hadamaard})
can be thought of as just a number expressed in binary. \ie,
\begin{equation}
\begin{split}
H\ket{\bf 0} =& \frac{1}{\sqrt{2^N}}\biggl\lbrace 
\underset{0}{\underbrace{\ket{0\ldots 0}}}
+ \underset{1}{\underbrace{\ket{0\ldots 01}}} + \cdots 
+ \underset{2^N-1}{\underbrace{\ket{1\ldots 1}}}\biggr\rbrace \\
=& \frac{1}{\sqrt{2^N}}\sum_{x=0}^{2^N-1} \ket{x}.
\end{split}
\end{equation}
This will be useful in function evaluation. 
\ie, Consider a function such as $f\colon\lbrace 0,1\rbrace^N\to\lbrace 
0,1\rbrace$.
Since the following are equivalent
\begin{equation}
\begin{split}
(0,0,0,\ldots,1,0)&\mapsto f(0,0,0,\ldots,1,0)\\
\underset{2}{\underbrace{(000\ldots 10)}}&
\mapsto f\underset{2}{\underbrace{(000\ldots 10)}}\\
(2)&\mapsto f(2),
\end{split}
\end{equation}
%Instead of thinking of this function as 
%\begin{equation}
%(0,0,0,\ldots,1,0)\mapsto f(0,0,0,\ldots,1,0),
%\end{equation}
one can simply think of this function as $x\mapsto f(x)$.
So, as a first step, the Hadamaard transformation acts on a quantum 
zero register, preparing it in a superposition of all possible values 
of $x$
\begin{equation}
H\ket{\bf 0} = \frac{1}{\sqrt{2^N}}\sum_{x=0}^{2^N-1} \ket{x}.
\end{equation}

A Hadamaard transformation acting on a {\sl general} state 
$\ket{x}=\ket{x_1\ldots x_N}$
(not necessarily all 0's) can be written as
\begin{equation}
\begin{split}
H\colon\ket{x}\quad\mapsto\quad&
\bigotimes_{i=1}^N 
H_1\ket{x_i}\\
\quad\mapsto\quad&
\bigotimes_{i=1}^N 
\frac{1}{\sqrt{2}} \left( \ket{0} + (-1)^{x_i}\ket{1} \right)\\
\quad\mapsto\quad&
\bigotimes_{i=1}^N 
\left(\frac{1}{\sqrt{2}} \sum_{y_i\in\lbrace 0,1\rbrace} 
(-1)^{x_iy_i}\ket{y_i}\right)\\
\quad\mapsto\quad&
\frac{1}{\sqrt{2^N}} \sum_{y=0}^{2^N-1} (-1)^{x\cdot y}\ket{y},
\end{split}
\label{e:generalHadamaard}
\end{equation}
where $x\cdot y$ denotes a dot product over $\left(\Z_2\right)^N$, or
simply a binary dot product.  Note that the final form of equation
(\ref{e:generalHadamaard}) is actually the Fourier transform of $\ket{x}$
over $\left(\Z_2\right)^N$.



%--------------------------------------------------------------------------------   
%--------------------------------------------------------------------------------   
\section{Quantum Algorithms}
\label{sec:quantCompAlgs}

Grover\cite{Grover:96}\dots did some stuff!

%--------------------------------------------------------------------------------   
%--------------------------------------------------------------------------------   
\section{Quantum Cryptography}

\subsubsection{QKDs \& proofs}

%--------------------------------------------------------------------------------   
%--------------------------------------------------------------------------------   
\section{Quantum Error Correction}

\subsubsection{ECCs}
\subsubsection{Code bdles?}
