
\chapter{Quantum Computation}
\label{chap:quantumComputation}
\index{Quantum Computation@\emph{Quantum Computation}}%

%--------------------------------------------------------------------------------   
%--------------------------------------------------------------------------------   
\section{Information Theory}

%--------------------------------------------------------------------------------   
\subsection{Classical}

%--------------------------------------------------------------------------------   
\subsection{Quantum}

%--------------------------------------------------------------------------------   
%--------------------------------------------------------------------------------   
\section{The Theory of Computation}

\mmm{Show circuits for some explicit computable function}

%--------------------------------------------------------------------------------   
\subsection{Classical}

%--------------------------------------------------------------------------------   
\subsection{Quantum}

\subsubsection{The Hadamaard Transformation}
\index{Hadamaard transformation}

The one--dimensional Hadamaard transformation takes a quantum bit into the
superposition given by
\begin{equation}
\begin{split}
\ket{0}\quad\mapsto&\quad\frac{1}{\sqrt{2}}\left(\ket{0} +\ket{1}\right)\\
\ket{1}\quad\mapsto&\quad\frac{1}{\sqrt{2}}\left(\ket{0} -\ket{1}\right),
\end{split}
\end{equation}
which is sometimes thought of in terms of matrices as
\begin{equation}
H_1 \left[\begin{matrix}\ket{0}\\ \ket{1}\end{matrix}\right]
= \frac{1}{\sqrt{2}}
\left[\begin{matrix}1&1\\ 1&-1\end{matrix}\right]
\left[\begin{matrix}\ket{0}\\ \ket{1}\end{matrix}\right].
\end{equation}

The version of the Hadamaard transformation that sees the most use is
the $N$--dimensional Hadamaard transformation defined as the $N$--fold
tensor product of one--dimensional Hadamaard transformations.
This is often used as the first step in many quantum algorithms.
Start with the \emph{zero register}
\index{zero register}
\begin{equation}
\ket{\bf 0}=\ket{0\ldots 0} = \bigotimes^N\ket{0},
\end{equation} and then act
\begin{equation}
\begin{split}
H\ket{\bf 0} =& H_N\ket{0\ldots 0}\\
=& \bigotimes^N H_1\ket{0}\\
=& \bigotimes^N \frac{1}{\sqrt{2}}\left(\ket{0} +\ket{1}\right)\\
=& \frac{1}{\sqrt{2^N}}\biggl\lbrace \ket{0\ldots 0} + \ket{0\ldots 01} +
\quad\cdots\quad + \ket{1\ldots 1}\biggr\rbrace.
\end{split}
\label{e:hadamaard}
\end{equation}
Now, each of the kets on the bottom--RHS of equation (\ref{e:hadamaard})
can be thought of as just a number expressed in binary. \ie,
\begin{equation}
\begin{split}
H\ket{\bf 0} =& \frac{1}{\sqrt{2^N}}\biggl\lbrace 
\underset{0}{\underbrace{\ket{0\ldots 0}}}
+ \underset{1}{\underbrace{\ket{0\ldots 01}}} + \cdots 
+ \underset{2^N-1}{\underbrace{\ket{1\ldots 1}}}\biggr\rbrace \\
=& \frac{1}{\sqrt{2^N}}\sum_{x=0}^{2^N-1} \ket{x}.
\end{split}
\end{equation}
This notation will be useful in function evaluation. 
\ie, Consider a function such as $f\colon\lbrace 0,1\rbrace^N\to\lbrace 
0,1\rbrace$.
Since the following are equivalent
\begin{equation}
\begin{split}
(0,0,0,\ldots,1,0)&\mapsto f(0,0,0,\ldots,1,0)\\
\underset{2}{\underbrace{(000\ldots 10)}}&
\mapsto f\underset{2}{\underbrace{(000\ldots 10)}}\\
(2)&\mapsto f(2),
\end{split}
\end{equation}
%Instead of thinking of this function as 
%\begin{equation}
%(0,0,0,\ldots,1,0)\mapsto f(0,0,0,\ldots,1,0),
%\end{equation}
one can simply think of this function as $x\mapsto f(x)$.








%--------------------------------------------------------------------------------   
%--------------------------------------------------------------------------------   
\section{Quantum Algorithms}


%--------------------------------------------------------------------------------   
%--------------------------------------------------------------------------------   
\section{Quantum Cryptography}

%--------------------------------------------------------------------------------   
%--------------------------------------------------------------------------------   
\section{Quantum Error Correction}
