
\chapter{Dynamical Stability}
\label{chap:dynamicalStability}
\index{Dynamical Stability@\emph{Dynamical Stability}}%


The primary model used in this dissertation approximates 
Grover's algorithm as a geodesic of the Fubini--Study metric 
as discussed in chapter \ref{chap:dynamics}.

%--------------------------------------------------------------------------------   
%--------------------------------------------------------------------------------   
\section{Linear Stability Analysis}

Inspection of the geodesic equations of motion 
(\ref{e:detailedGeodesics}) yields a fixed point about the 
origin.  No other fixed points are expected. \mmm{Why?}
The Jacobian of the evolution vanishes near the origin, and
so the equations of motion of the linearized problem become
the trivial geodesic equations for a flat metric.
This result is not particularly useful in characterizing how 
the algorithm will behave, at finite step size, in a noisy 
environment.  
A method of stability analysis
is therefore needed that does not rely upon a linearization
of the problem.  This gets quite difficult to do in general,
so a numerical approach is then warranted.

%--------------------------------------------------------------------------------   
%--------------------------------------------------------------------------------   
\section{Numerical Model of Grover's Algorithm}

A database consisting of $N=2^n$ items, is represented (using
$n$ qubits) as an $N$--state system whose states are given as
points in $\CP^{N-1}$.  The evolution of the algorithm is then 
modelled by the numerical integration of the geodesic equations 
of motion (\ref{e:detailedGeodesics}) on this space.
Appropriate initial conditions are supplied to mimic the behaviour of 
the actual algorithm.  At this stage, the approach involves only pure 
states, and so the (unitary) evolution remains on the shell of pure 
states.
\mmm{show baseline results here?}
\mmm{uncertainties here?}
\mmm{pseudotime relates to number of Grover iterations}

%--------------------------------------------------------------------------------   
\subsection{Simulation of Grover as a dynamical map}

The time evolution of a quantum mechanical system is most easily
described by unitary evolution.
Unfortunatley, physical systems do not exist in isolation.  There
are always interactions with environments or baths that cause
quantum mechanical systems to evolve by possibly more complicated
means than these simple unitary transformations.

A somewhat more general approach to quantum evolution is to 
consider dynamical evolution of a quantum mechanical system
described as simply a map, a \emph{dynamical map}\cite{Sudarshan:??}
of density matrices to density matrices.
Notice that an even more general approach is to look at maps of 
density matrices, where the image itself doesn't necessarily have to
be a physical density matrix.  It could be simply a subsystem
of a physical system, and hence it's trace could be less than
one.  For purposes of this dissertation, only dynamical maps that
take physical density matrices to physical density matrices are
considered.

\mmm{Phase space stuff or configuration space stuff}

The numerical simulation of Grover's algorithm described above
integrates the geodesic equations of motion.  At every time step,
this numerical integration can be viewed as simply a map that
takes density matrices to density matrices\dots a dynamical map.
\footnote{Really, this particular mapping maps derivatives as well, and
can be thought of as a mapping of the state in a quantum phase 
space.}

%--------------------------------------------------------------------------------   
%--------------------------------------------------------------------------------   
\section{Noise}

Now, any noise introduced into the dynamical evolution of a 
quantum mechanical system can be described within the context 
of dynamical maps.  
No matter the source of the noise or whether the noise is 
Hamiltonian or non-Hamiltonian, this approach will encompass
it all.
\mmm{Include simple analytical perturbation if possible.  Use this to
compare with code}
\mmm{maybe example with simple problem as dynamical map...add noise
(physically motivated noise)}

Try two different models\dots
\begin{enumerate}
\item Run density matrices.  Take finite Grover iterations, then add
infinitesimal random Hermitian matrices.  This will model some general
noise in Grover(?). This will take pure to mixed states.
\item Represent mixed state density matrices as
\begin{equation}
\rho = \sum\lambda_i\rho_i
\end{equation}
and let the $\rho_i$ flow as pure states.  Add in noise to $\lambda$s
{\sl and} to the $\rho_i$ evolutions.
\end{enumerate}

\subsubsection{No Noise}
\subsubsection{...}
\subsubsection{Max Noise}

