
\chapter{Dynamical Stability}
\label{chap:dynamicalStability}
\index{Dynamical Stability@\emph{Dynamical Stability}}%



%--------------------------------------------------------------------------------   
%--------------------------------------------------------------------------------   
\section{Linear Stability Analysis}

About the origin, the geodesic equations become those of a flat metric.
Not useful, need a method that works even without a linearization.
Gets hard\dots do numerical.

\mmm{put details of this in?}

%--------------------------------------------------------------------------------   
%--------------------------------------------------------------------------------   
\section{Numerical Model of Grover's Algorithm}

The primary model used in this dissertation uses the fact
that Grover can be approximated by a geodesic of the
Fubini--Study metric as discussed in chapter \ref{chap:dynamics},

A database consisting of $N=2^n$ items, is represented (using
$n$ qubits) as an $N$--state system.
The numerical model represents the states of this system as points 
in $\CP^{N-1}$.  The evolution of the algorithm is then modelled by the
numerical integration of the geodesic equations of motion
(\ref{e:detailedGeodesics}), with 
appropriate initial conditions supplied to mimic the behaviour of 
the algorithm.  At this stage, this approach involves only pure 
states, and the (unitary) evolution remains on this shell of pure 
states.
\mmm{show baseline results here?}
\mmm{uncertainties here?}
\mmm{pseudotime relates to number of Grover iterations}


%--------------------------------------------------------------------------------   
%\subsection{Dynamical Maps}
Open systems.
What the hell are dynamical maps?

Unfortunatley, physical systems do not exist in isolation.  There
are always interactions with environments or baths that cause
quantum mechanical systems to evolve by possibly more complicated
means than unitary transformations.

A somewhat more general approach to quantum evolution is to 
consider dynamical evolution of a quantum mechanical system
described as simply a map of density matrices to density
matrices, the so--called \emph{dynamical map}.

An even more general approach is to simply look at maps of
density matrices, where the image itself doesn't necessarily have to
be a physical density matrix.  It could be simply a subsystem
of a physical system, and hence it's trace could be less than
one.

\mmm{Phase space stuff or configuration space stuff}

\subsubsection{Simulation of Grover is a dynamical map}




%--------------------------------------------------------------------------------   
%--------------------------------------------------------------------------------   
\section{Noise}

Don't worry about Hamiltonian or non-Hamiltonian noise, this will encompass
them both.

How to characterize noise?

First try some simple examples\dots
\ie, harmonic oscillator coupled to another harmonic oscillator.
Evolve, then trace out one of the oscillators.  Is this the same as
a time--dependent perturbation of a harmonic oscillator, where the
perturbation itself is also an oscillator?  How should I couple these
systems?  Try just $\gamma\left( a b^\dagger + b a^\dagger\right)$?
Or maybe something like $\gamma\left| x-y\right|$?

\mmm{Try using some dynamical systems techniques and tools to do this}

%--------------------------------------------------------------------------------   
\subsection{Some Physical Mechanisms for Noise}

\subsubsection{No Noise}
\subsubsection{...}
\subsubsection{Max Noise}

\mmm{Include simple analytical perturbation if possible.  Use this to
compare with code}

How does noise in gates propagate out?

Can anything be said about chaos?

How does this compare with bit--flips?

Try two different models\dots
\begin{enumerate}
\item Run density matrices.  Take finite Grover iterations, then add
infinitesimal random Hermitian matrices.  This will model some general
noise in Grover(?). This will take pure to mixed states.
\mmm{Do mixed states really make sense for Grover?}
\item Represent mixed state density matrices as
\begin{equation}
\rho = \sum\lambda_i\rho_i
\end{equation}
and let the $\rho_i$ flow as pure states.  Add in noise to $\lambda$s
{\sl and} to the $\rho_i$ evolutions.
\end{enumerate}

