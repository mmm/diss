
\chapter{Dynamical Stability}
\label{chap:dynamicalStability}
\index{Dynamical Stability@\emph{Dynamical Stability}}%



Don't worry about Hamiltonian or non-Hamiltonian noise, this will encompass
them both.

%--------------------------------------------------------------------------------   
%--------------------------------------------------------------------------------   
\section{Noise}

How to characterize noise?

First try some simple examples\dots
\ie, harmonic oscillator coupled to another harmonic oscillator.
Evolve, then trace out one of the oscillators.  Is this the same as
a time--dependent perturbation of a harmonic oscillator, where the
perturbation itself is also an oscillator?  How should I couple these
systems?  Try just $\gamma\left( a b^\dagger + b a^\dagger\right)$?
Or maybe something like $\gamma\left| x-y\right|$?

\mmm{Try using some dynamical systems techniques and tools to do this}

%--------------------------------------------------------------------------------   
\subsection{Some Physical Mechanisms for Noise}

\subsubsection{No Noise}
\subsubsection{...}
\subsubsection{Max Noise}

\mmm{Include simple analytical perturbation if possible.  Use this to
compare with code}

How does noise in gates propagate out?

Can anything be said about chaos?

How does this compare with bit--flips?

Try two different models\dots
\begin{enumerate}
\item Run density matrices.  Take finite Grover iterations, then add
infinitesimal random Hermitian matrices.  This will model some general
noise in Grover(?). This will take pure to mixed states.
\mmm{Do mixed states really make sense for Grover?}
\item Represent mixed state density matrices as
\begin{equation}
\rho = \sum\lambda_i\rho_i
\end{equation}
and let the $\rho_i$ flow as pure states.  Add in noise to $\lambda$s
{\sl and} to the $\rho_i$ evolutions.
\end{enumerate}



%--------------------------------------------------------------------------------   
%--------------------------------------------------------------------------------   
\section{Stability Analysis}

%--------------------------------------------------------------------------------   
\subsection{Perturbations}

%--------------------------------------------------------------------------------   
\subsection{Linear Stability Analysis}

About the origin, the geodesic equations become those of a flat metric.
Not useful, need a method that works even without a linearization.

\mmm{put this in?}

%--------------------------------------------------------------------------------   
\subsection{Finite--Time Lyapunov Exponents}

