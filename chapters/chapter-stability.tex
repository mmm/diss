
\chapter{Dynamical Stability}
\label{chap:dynamicalStability}
\index{Dynamical Stability@\emph{Dynamical Stability}}%




%--------------------------------------------------------------------------------   
%--------------------------------------------------------------------------------   
\section{Noise}

How to characterize noise?

First try some simple examples\dots
\ie, harmonic oscillator coupled to another harmonic oscillator.
Evolve, then trace out one of the oscillators.  Is this the same as
a time--dependent perturbation of a harmonic oscillator, where the
perturbation itself is also an oscillator?  How should I couple these
systems?  Try just $\gamma\left( a b^\dagger + b a^\dagger\right)$?
Or maybe something like $\gamma\left| x-y\right|$?

\mmm{Try using some dynamical systems techniques and tools to do this}

%--------------------------------------------------------------------------------   
\subsection{Some Physical Mechanisms for Noise}

\subsubsection{No Noise}
\subsubsection{...}
\subsubsection{Max Noise}

\mmm{Include simple analytical perturbation if possible.  Use this to
compare with code}



%--------------------------------------------------------------------------------   
%--------------------------------------------------------------------------------   
\section{Stability Analysis}

%--------------------------------------------------------------------------------   
\subsection{Perturbations}

%--------------------------------------------------------------------------------   
\subsection{Linear Stability Analysis}

About the origin, the geodesic equations become those of a flat metric.
Not useful, need a method that works even without a linearization.

%--------------------------------------------------------------------------------   
\subsection{Finite--Time Lyapunov Exponents}

