\chapter{Quantum Geometry}
\label{chap:quantumGeometry}
\index{Quantum Geometry@\emph{Quantum Geometry}}

%---------------------------------------------------------------------------------------
%---------------------------------------------------------------------------------------
\section{Statistical Distance and Hilbert Space}
\label{sec:statisticalDistance}
\index{Statistical Distance@\emph{Statistical Distance}}%


In classical probability, the \emph{distance} between two distinguishable 
events in state space can be characterized in terms of the maximum number
of mutually distinguishable intermediate events.  This notion was 
generalized to quantum states by Wooters\cite{Wooters:81}, who found
that the \emph{statistical distance} between two quantum states coincides with
the \emph{metric distance} between two rays in Hilbert space.

Optimization of the statistical distinguishability of quantum
states leads to a natural Hermitian metric on the space of density
operators\cite{Braunstein/Caves:94}.  This metric is, for pure states,
the usual Fubini--Study metric on projective Hilbert space\cite{Gibbons:92}.
It can be extended to general density operators\cite{Braunstein/Caves:94}.

%--------------------------------------------------------------------------------   
\subsection{Classical Statistical Distance}

It is worth noting, that statistical distance on a classical state space 
is not just the Euclidean distance between two probabilities $|p_1-p_2|$.  
This is due to the fact that probabilities near $\frac{1}{2}$ are harder 
to distinguish due to higher statistical fluctuations than probabilities 
near 0 or 1.

As an example, consider two differently weighted coins.  Coin $i$ has
an {\it a priori} probability $p_i$ of obtaining heads in a given toss.
The probability space is one dimensional, just the probability 
$p_i$ of heads for each coin.
Define, following Wooters\cite{Wooters:81},
the statistical distance between two coins in this probability space as
\begin{equation}
d(p_1,p_2) = \lim_{n\to\infty} \frac{1}{\sqrt{n}}\times
    \left[\begin{matrix}
        \text{\scriptsize maximum number of mutually distinguishable}\\
        \text{\scriptsize (in {\it n} trials) intermediate probabilities}
    \end{matrix}\right],
\label{e:ClassicalDistance}
\end{equation}
where two intermediate probabilities $p$ and $p^\prime$ are distinguishable
in $n$ trials if 
\begin{equation}
\left| p-p^\prime \right| \ge \Delta p + \Delta p^\prime.
\end{equation}
An expression for $\Delta p$ can be given in the spirit of experimental
uncertainty as RMS deviation
\begin{equation}
\Delta p = \left[ \frac{p(1-p)}{n} \right]^\frac{1}{2}.
\end{equation}
This gives 
\begin{equation}
\begin{split}
d(p_1,p_2) =& \lim_{n\to\infty} \frac{1}{\sqrt{n}}\times
\int_{p_1}^{p_2}\,\frac{dp}{2\Delta p}\\
=& \lim_{n\to\infty} \frac{1}{\sqrt{n}}\times
\int_{p_1}^{p_2}\,\frac{\sqrt{n}dp}{2\sqrt{p(1-p)}}\\
=& \int_{p_1}^{p_2}\,\frac{dp}{2\sqrt{p(1-p)}}\\
=& \cos^{-1}\left( 
    \sqrt{p_1} \sqrt{p_2} 
    +\sqrt{q_1}\sqrt{q_2}
\right),
\end{split}
\end{equation}
where $q_1 = 1-p_1$ and $q_2=1-p_2$.  So, the statistical distance
is not the Euclidean distance ($\left| p_2 - p_1 \right|$) on probability
space.
It is easy to generalize this distance when the system has 
more that just two outcomes, $p$ and $q$.
When there are countably infinite outcomes, the notion of statistical
distance becomes
\begin{equation}
d(p_1,p_2) = \cos^{-1}\left\lbrack 
    \sum_{i=1}^\infty
    \sqrt{p_1} \sqrt{p_2} 
\right\rbrack.
\end{equation}




%--------------------------------------------------------------------------------   
\subsection{A Quantum Mechanical Approach}
\label{ssec:qmapproach}


Consider two pure states:
\begin{equation}
\begin{split}
    \ket{\psi} =& \sum_j \sqrt{p_j}e^{i\phi_j}\ket{j},\\
    \ket{\tilde\psi} = \ket{\psi} + \ket{d\psi} =& 
              \sum_j \sqrt{p_j + dp_j}e^{i(\phi_j + d\phi_j)}\ket{j}.
\end{split}
\label{e:purestates}
\end{equation}
For $\ket{\psi}$ and $\ket{\tilde\psi}$ normalized, we expect
\begin{equation}
\braket{\psi}{\tilde\psi} = cos(\theta),
\end{equation}
where $\theta$ is the ``angle'' between these two states in 
(projective) Hilbert space.  This angle presents a measure of the
notion of state distinguishability.  \ie,
\begin{equation}
\frac{1}{4}ds_{\text{\tiny Pure States}}^2 = 
    \left[ \cos^{-1}\left(\left|
                        \braket{\tilde\psi}{\psi}
                    \right|\right)
    \right]^2
= 1-\left| \braket{\tilde\psi}{\psi} \right|^2    
= \braket{d\psi_\perp}{d\psi_\perp}.
\end{equation}
If we define $\ket{d\psi_\perp} = \ket{d\psi} - \ket{\psi}\braket{\psi}{d\psi}$, 
then
\begin{equation}
    \braket{d\psi_\perp}{d\psi_\perp} = 
        \braket{d\psi}{d\psi} - \left| \braket{\psi}{d\psi} \right|^2.
\label{e:dPsi_perp}
\end{equation}
Since (from (\ref{e:purestates})) 
$\ket{d\psi} = \ket{\tilde\psi} - \ket{\psi}$, we can write
\begin{equation}
\begin{split}
    \braket{d\psi}{d\psi} 
        =& \left(\bra{\tilde\psi}-\bra{\psi}\right)
           \left(\ket{\tilde\psi}-\ket{\psi}\right)\\
        =& \braket{\tilde\psi}{\tilde\psi} - \braket{\tilde\psi}{\psi}
          - \braket{\psi}{\tilde\psi} + \braket{\psi}{\psi}\\
        =& \,2 - \braket{\psi}{\tilde\psi} - \braket{\tilde\psi}{\psi}\\
        =& \,2 - \left(
                      \sum_j\sqrt{p_j}\sqrt{p_j+dp_j}e^{id\phi_j}
                \right) 
                - \left(
                      \sum_j\sqrt{p_j}\sqrt{p_j+dp_j}e^{-id\phi_j}
                \right)\\
        =& \,2 - 2 \sum_j\sqrt{p_j}\sqrt{p_j+dp_j}
                \,\cos(d\phi_j).
\end{split}
\end{equation}
Now, we can expand to second order in $d\phi$ and $dp$ to get
\begin{equation}
\begin{split}
    \braket{d\psi}{d\psi}
        \approx& \,2\left\lbrace1 - \sum_jp_j\left(1+\frac{dp_j}{p_j}\right)^\frac{1}{2} 
                    \left[ 1 - \frac{(d\phi_j)^2}{2}\right]
            \right\rbrace\\
        \approx& \,2\left\lbrace1 - \sum_jp_j
                    \left[1+\frac{1}{2}\left(\frac{dp_j}{p_j}\right)
                           -\frac{1}{8}\left(\frac{dp_j}{p_j}\right)^2\right]
                    \left[ 1 - \frac{(d\phi_j)^2}{2}\right]
            \right\rbrace\\
        =& \,2\left\lbrace1 - \sum_j
                    \left( p_j + \frac{1}{2}dp_j 
                           -\frac{1}{8}\frac{(dp_j)^2}{p_j}
                           -\frac{1}{2}p_j(d\phi_j)^2
                    \right)
            \right\rbrace.\\
\end{split}
\end{equation}
Now, $\lbrace p_i\rbrace$ are probabilities, so $\sum_jp_j=1$ and $\sum_jdp_j = 0$,
which gives
\begin{equation}
    \braket{d\psi}{d\psi}
        \approx \sum_j 
                    \left( 
                           \frac{1}{4}\frac{(dp_j)^2}{p_j}
                           +p_j(d\phi_j)^2
                    \right).
\end{equation}
%
Similarly, 
\begin{equation}
\begin{split}
    \braket{\psi}{d\psi}
        =& \bra{\psi} \left(\ket{\tilde\psi} - \ket{\psi}\right)
        = \braket{\psi}{\tilde\psi} - \braket{\psi}{\psi}\\
        =& \braket{\psi}{\tilde\psi} - 1\\
        =& \sum_j\sqrt{p_j}\sqrt{p_j+dp_j}\,e^{id\phi_j} - 1.
\end{split}
\end{equation}
Expanding to first order in $dp$ and $d\phi$, we get
\begin{equation}
\begin{split}
    \braket{\psi}{d\psi}
        \approx& \left\lbrace \sum_jp_j
                    \left[1+\frac{1}{2}\left(\frac{dp_j}{p_j}\right)
                           %-\frac{1}{8}\left(\frac{dp_j}{p_j}\right)^2
                    \right]
                    \left[ 1 + id\phi_j 
                            %- \frac{(d\phi_j)^2}{2}
                    \right]
                 \right\rbrace
            - 1\\ 
        \approx& \sum_j \left( p_j + \frac{1}{2}dp_j + ip_jd\phi_j
                  \right) -1\\
        =&\, i \sum_j p_jd\phi_j,
\end{split}
\end{equation}
and then equation (\ref{e:dPsi_perp}) gives
\begin{equation}
\begin{split}
    \braket{d\psi_\perp}{d\psi_\perp} 
        =&\, \braket{d\psi}{d\psi} - \left| \braket{\psi}{d\psi} \right|^2.\\
        \approx& \sum_j 
                    \left( 
                           \frac{1}{4}\frac{(dp_j)^2}{p_j}
                           +p_j(d\phi_j)^2
                    \right)
            - \left( \sum_jp_jd\phi_j \right)^2\\
        \approx& \sum_j \frac{1}{4}\frac{(dp_j)^2}{p_j}
                  + \left[ 
                           \sum_j p_j(d\phi_j)^2
                           - \left( \sum_jp_jd\phi_j \right)^2
                    \right].
\end{split}
\end{equation}

\mmm{fix this}
I'd like to characterize ``Statistical Distance'' as defined by Wooters\cite{Wooters:81}
geometrically.  I'll follow Braunstein and Caves\cite{Braunstein/Caves:94}
to get the Fubini--Study metric for projective Hilbert space\dots
\begin{equation}
                ds_{\text{SP}}^2 = \sum_i \frac{1}{4}\frac{(dp_i)^2}{p_i} + 
                \left( 
                    \sum_i p_i(d\phi_i)^2 - \left( \sum_i p_i d\phi_i \right)^2
                \right).
\label{e:fsStatistical}
\end{equation}

%--------------------------------------------------------------------------------   
%--------------------------------------------------------------------------------   
\section{The Geometry of Quantum Mechanics}
\label{sec:geometryQM}
\index{Quantum Geometry}

%--------------------------------------------------------------------------------   
\subsection{Complex Projective Space}
\index{complex!projective space}

The complex projective space $\CP^n$ is the set of complex lines
through the origin of $\C^{n+1}$.
A point $z=(z^0,\ldots,z^n)\in\C^{n+1}$ defines a line through
the origin provided $z\ne 0$.\index{complex!line}
Another point $w\in\C^{n+1}$ lies on the same line as that
defined by $z$ if there exists a complex number $\alpha\ne 0$
such that $w=\alpha z$.  This defines an equivalence relation $\sim$.
\ie, $w\sim z$ if there exists $\alpha\in\C^{n+1}\setminus\lbrace 0\rbrace$
such that $w=\alpha z$.  Then, $\CP^n$ can be defined as 
$\CP^n=\left( \C^{n+1}\setminus\lbrace 0\rbrace\right)/\sim$.

The set of $n+1$ coordinates $z^0,\ldots,z^n$ are referred to as
\emph{homogeneous coordinates}
\index{complex!projective space!homogeneous coordinates}
for $\CP^n$.  However, $\CP^n$ is an $n$-dimensional complex
manifold with ($n$) coordinates defined as follows.
Consider the coordinate patch $U_i$ defined as the set of all
complex lines through the origin of $\C^{n+1}$ such that $z^i\ne 0$.
Define coordinates on this patch by
\begin{equation}
\xi^j_{(i)} = \frac{z^j}{z^i}.
\end{equation}
These are \emph{inhomogeneous coordinates} for $\CP^n$.
\index{complex!projective space!inhomogeneous coordinates}

For a point $z=(z^0,\ldots,z^n)$ in the intersection of 
two coordinate patches $U_i\cap U_j$,
the two inhomogeneous coordinates for this point
$\xi^k_{(i)}={z^k}/{z^i}$ and
$\xi^k_{(j)}={z^k}/{z^j}$
are related by the transition functions
\begin{equation}
\psi_{ij}\colon\quad \xi^k_{(j)}\quad\mapsto
\quad\xi^k_{(i)} = \left( \frac{z^j}{z^i} \right)
\xi^k_{(j)},
\end{equation}
which makes $\CP^n$ a nice smooth complex manifold.

Inhomogeneous coordinates for $\CP^n$ will be used
exclusively throughout the remainder of this dissertation.
Consequently, I will adapt a slight abuse of notation to simplify
many later expressions.
\index{notation}
The $n+1$ \emph{homogeneous} coordinates for a point in $\C^{n+1}$ will be expressed
as $(z^0,z^1,\ldots,z^n)$, whereas the $n$ \emph{inhomogeneous} coordinates for the
corresponding point in $\CP^n$ will be expressed, not as
$(\xi^1_{(0)},\ldots,\xi^n_{(0)})$ in the $U_0$ coordinate patch, but
simply as $(z^1,\ldots,z^n)$, with the coordinate patch implied.
Nothing will be covered that is dependent upon which 
particular coordinate patch
is used.

For more on complex manifolds and projective spaces
see Nash\cite{Nash:91} and Nakahara\cite{Nakahara:90}.



%--------------------------------------------------------------------------------   
\subsection{The Fubini--Study Metric}
%\index{Complex Projective Space!Fubini--Study Metric}
\index{complex!projective space!Fubini--Study metric}

%For any two vectors $X,Y\in T_p\C P^n$, we can define
%$g\colon T_p\C P^n \to T_p\C P^n$ by
%\begin{equation}
%g(X,Y) = \Omega(X,JY),
%\end{equation}
%where $\Omega$ is the K\"ahler form and $J$ is the complex structure.
%
A natural Hermitian metric can be defined on complex projective
space.\cite{Fubini:?,Study:?}
\mmm{How is it natural?}
The Fubini--Study metric can be given, in inhomogeneous coordinates, as
\begin{equation}
ds_{FS}^2 = g_{\overline{i}j}\, dz^i\otimes d\overline{z}^j,
\end{equation}
where 
\begin{equation}
g_{\overline{i}j} = 4\,\frac{\delta_{ij}\left( 1 + \left| z \right|^2 \right) - 
                                                    \overline{z}_i z_j }
{ \left( 1 + \left| z\right|^2 \right)^2 }.
\label{e:fsmetric}
\end{equation}


\subsubsection{Fubini--Study as Statistical Distance}

The Fubini--Study metric is a notion of distance on a projective
Hilbert space that corresponds exactly to expressions of statistical
distance previously discussed.  To see this, consider the 
line element from the Fubini--Study
metric 
(equation (\ref{e:fsmetric})) 
\begin{equation}
\begin{split}
ds^2 =& \biggl(
            \frac{\delta_{ij}\left( 1 + |z|^2 \right) - \overline{z}_iz_j}
                 { \left( 1 + |z|^2 \right)^2 }
       \biggr)
       dz^id\overline{z}^j\\
=& \frac{ dz^id\overline{z}^i }{ \left( 1 + |z|^2 \right) }
- \frac{ \overline{z}_idz^iz_jd\overline{z}^j }
                 { \left( 1 + |z|^2 \right)^2 }.
\end{split}
\end{equation}
In polar coordinates
\begin{equation}
\begin{split}
z_j=&\,r_je^{i\phi_j}\\
\overline{z}_j=&\,r_je^{-i\phi_j},
\end{split}
\end{equation}
with derivatives
\begin{equation}
\begin{split}
dz_j =& \left( dr_j + ir_jd\phi_j \right) e^{i\phi_j}\\
d\overline{z}_j =& \left( dr_j - ir_jd\phi_j \right) e^{-i\phi_j},
\end{split}
\end{equation}
this metric looks like
\begin{multline}
ds^2 = \frac{1}{\left( 1+r^2 \right)}
    \biggl[\sum_i\left(dr_i\right)^2 + \sum_i r_i^2\left(d\phi_i\right)^2\biggr]\\
- \frac{1}{\left( 1+r^2 \right)^2}
    \biggl[\biggl(\sum_i r_idr_i\biggr)^2
           + \biggl(\sum_i r_i^2d\phi_i\biggr)^2
    \biggr].
\label{e:fsPolar}
\end{multline}

Now, to see the connection with the statistical distance described
in section (\ref{ssec:qmapproach}),
consider coordinates for an $(n+1)$--state system
\begin{equation}
\ket{\Psi} = a_0\ket{0} + \cdots + a_n\ket{n}
\end{equation} 
with
\begin{equation}
a_i = \sqrt{p_i}e^{i\phi_i}\qquad i=0\ldots n+1.
\end{equation}
We note here that the $p_i$ are probabilities and hence obey
\begin{equation}
\begin{split}
\sum_{i=0}^{n} p_i =& 1\\
\sum_{i=0}^{n} dp_i =& 0.
\end{split}
\end{equation}
If these coordinates are viewed as $n+1$ \emph{homogeneous}
coordinates to $\CP^{n}$, then the $n$ corresponding 
\emph{inhomogeneous} coordinates can then be obtained
(in coordinate patch $U_0$, where $a_0\ne 0$) as
\begin{equation}
z_i = \frac{a_i}{a_0} = \frac{ \sqrt{p_i}e^{i\phi_i} }
                             { \sqrt{p_0}e^{i\phi_0} }
= \sqrt{\frac{p_i}{p_0}} e^{i(\phi_i - \phi_0)}.
\label{e:inhomoA}
\end{equation}
Physically, freedom of normalization and overall phase allow
\begin{equation} 
\ket{\Psi} =\, a_0\left( \ket{0} + \frac{a_1}{a_0}\ket{1} + \cdots 
                        + \frac{a_n}{a_0}\ket{n}
                \right)
\end{equation}
to be written simply as
\begin{equation}
\ket\Psi  =\, \ket{0} + z_1\ket{1} + \cdots + z_n\ket{n},
\end{equation}
with
\begin{equation}
z_i = \sqrt{\frac{p_i}{p_0}} e^{i\phi_i}.
\label{e:inhomoB}
\end{equation}
%A normalized version of $\ket{\Psi}$ would, of course, look like
%\begin{equation}
%\ket\Psi =\, \frac{1}{1+|z|^2}\biggl( \ket{0} + z_1\ket{1} + \cdots + z_n\ket{n}
%                        \biggr). 
%\end{equation}

The polar form of the Fubini--Study metric (\ref{e:fsPolar}),
can now be written in terms of probabilities $p_i$ to obtain 
(\ref{e:fsStatistical}),
the expected expression for statistical distance.
If 
\begin{equation}
r_i = \sqrt{\frac{p_i}{p_0}}\qquad(i= 1\ldots n),
\end{equation}
then 
\begin{equation}
r^2 = \sum_{i=1}^{n} r_i^2 = \sum_{i=1}^{n} \frac{p_i}{p_0}
= \frac{1}{p_0}\sum_{i=1}^n p_i.
\end{equation}
Now, since $\sum_{i=0}^n p_i = 1$,
\begin{equation}
\begin{split}
\sum_{i=1}^n p_i =& 1-p_0\\
\sum_{i=1}^n dp_i =& -dp_0.
\end{split}
\end{equation}
Then,
\begin{equation}
r^2 = \frac{1}{p_0} - 1,
\end{equation}
and the factor of $1+r^2$ in (\ref{e:fsPolar}) 
becomes simply $\frac{1}{p_0}$.
The Fubini--Study metric then becomes
\begin{equation}
ds^2 = p_0 \biggl[\sum_i \left(dr_i\right)^2 + \sum_i r_i^2\left(d\phi_i\right)^2\biggr]
- p_0^2 \biggl[\biggl(\sum_i r_idr_i\biggr)^2
           + \biggl(\sum_i r_i^2d\phi_i\biggr)^2
    \biggr].
\label{e:fsPolarHalfA}
\end{equation}
Now, since
\begin{equation}
dr_i = \frac{1}{2}\left( p_i^{-\frac{1}{2}}p_0^{-\frac{1}{2}} dp_i
                        -p_i^{\frac{1}{2}}p_0^{-\frac{3}{2}} dp_0
                  \right)
\end{equation}
leads to  
\begin{equation}
(dr_i)^2 = \frac{(dp_i)^2}{4p_ip_0}
            + \frac{p_i(dp_0)^2}{4p_0^3}
            - \frac{dp_idp_0}{2p_0^2}
\end{equation}
and
\begin{equation}
r_idr_i = \frac{dp_i}{2p_0} - \frac{p_idp_0}{2p_0^2},
\end{equation}
then (\ref{e:fsPolarHalfA}) becomes
\begin{multline}
ds^2 = p_0 \biggl[
    \sum_i\left( \frac{(dp_i)^2}{4p_ip_0}
                 + \frac{p_i(dp_0)^2}{4p_0^3}
                 - \frac{dp_idp_0}{2p_0^2}
          \right) 
    + \frac{1}{p_0} \sum_i p_i\left(d\phi_i\right)^2\biggr]\\
- p_0^2 \biggl[\biggl(\sum_i\frac{dp_i}{2p_0} 
                      - \sum_i\frac{p_idp_0}{2p_0^2}
               \biggr)^2
           + \frac{1}{p_0^2}\biggl(\sum_i p_id\phi_i\biggr)^2
    \biggr].
\end{multline}
Now, the expression for $r_idr_i$ simplifies somewhat
\begin{equation}
\begin{split}
\sum_i r_idr_i =& \sum_i\frac{dp_i}{2p_0} - \sum_i\frac{p_idp_0}{2p_0^2}\\
               =& \frac{1}{2p_0}\left(\sum_i dp_i\right)
                  - \frac{dp_0}{2p_0^2}\left(\sum_i p_i\right)\\
               =& \frac{1}{2p_0}\left( -dp_0 \right)
                  - \frac{dp_0}{2p_0^2}\left( 1 - p_0 \right)\\
               =& - \frac{dp_0}{2p_0} - \frac{dp_0}{2p_0^2} + \frac{dp_0}{2p_0}\\
               =&  - \frac{dp_0}{2p_0^2}.
\end{split}
\end{equation}
The line element then becomes
\begin{multline}
ds^2 = p_0 \biggl[
    \sum_i\left( \frac{(dp_i)^2}{4p_ip_0}
                 + \frac{p_i(dp_0)^2}{4p_0^3}
                 - \frac{dp_idp_0}{2p_0^2}
          \right) 
    + \frac{1}{p_0} \sum_i p_i\left(d\phi_i\right)^2\biggr]\\
- p_0^2 \biggl[\biggl(
                      - \frac{dp_0}{2p_0^2}
               \biggr)^2
           + \frac{1}{p_0^2}\biggl(\sum_i p_id\phi_i\biggr)^2
    \biggr],
\end{multline}
or
\begin{equation}
ds^2 = \sum_i\left( \frac{(dp_i)^2}{4p_i}
                 + \frac{p_i(dp_0)^2}{4p_0^2}
                 - \frac{dp_idp_0}{2p_0}
             \right) 
    + \sum_i p_i\left(d\phi_i\right)^2
    - \frac{dp_0^2}{4p_0^2}
    - \biggl(\sum_i p_id\phi_i\biggr)^2.
\label{e:fsPolarHalfC}
\end{equation}
The first sum can be reduced
\begin{equation}
\begin{split}
&\sum_i\left( \frac{(dp_i)^2}{4p_i}
             + \frac{p_i(dp_0)^2}{4p_0^2}
             - \frac{dp_idp_0}{2p_0}
      \right) \\
&= \sum_i\frac{1}{4}\frac{(dp_i)^2}{p_i}
   + \frac{dp_0^2}{4p_0^2}\left(\sum_i p_i\right)
   - \frac{dp_0}{2p_0}\left( \sum_i dp_i \right)\\
&= \sum_i\frac{1}{4}\frac{(dp_i)^2}{p_i}
   + \frac{dp_0^2}{4p_0^2}\left( 1 - p_0 \right)
   - \frac{dp_0}{2p_0}\left( -dp_0 \right)\\
&= \sum_i\frac{1}{4}\frac{(dp_i)^2}{p_i}
   + \frac{dp_0^2}{4p_0^2}
   + \frac{dp_0^2}{4p_0},
\end{split}
\end{equation}
and then (\ref{e:fsPolarHalfC}) becomes
\begin{equation}
\begin{split}
ds^2 =& 
    \left(
        \sum_i\frac{1}{4}\frac{(dp_i)^2}{p_i}
        + \frac{dp_0^2}{4p_0^2}
        + \frac{dp_0^2}{4p_0}
    \right)
    + \sum_i p_i\left(d\phi_i\right)^2
    - \frac{dp_0^2}{4p_0^2}
    - \biggl(\sum_i p_id\phi_i\biggr)^2\\
  =& 
    \sum_{i=1}^n\frac{1}{4}\frac{(dp_i)^2}{p_i}
    + \frac{1}{4}\frac{dp_0^2}{p_0}
    + \sum_{i=1}^n p_i\left(d\phi_i\right)^2
    - \biggl(\sum_{i=1}^n p_id\phi_i\biggr)^2\\
  =& 
    \sum_{i=0}^n\frac{1}{4}\frac{(dp_i)^2}{p_i}
    + \sum_{i=1}^n p_i\left(d\phi_i\right)^2
    - \biggl(\sum_{i=1}^n p_id\phi_i\biggr)^2.
\end{split}
\label{e:fsPolarFinal}
\end{equation}
Now, since the overall phase $\phi_0$ was chosen to remain 
zero when specifying inhomogeneous coordinates ((\ref{e:inhomoA})
to (\ref{e:inhomoB})), then $d\phi_0 = 0$, and all sums in 
(\ref{e:fsPolarFinal}) can be extended to include the zero case.

So then,
\begin{equation}
ds^2 = \sum_{i=0}^n\frac{1}{4}\frac{(dp_i)^2}{p_i}
    + \sum_{i=0}^n p_i\left(d\phi_i\right)^2
    - \biggl(\sum_{i=0}^n p_id\phi_i\biggr)^2
\end{equation}
matches with the statistical distance (\ref{e:fsStatistical}).

%--------------------------------------------------------------------------------   
\subsection{Geodesics}
\index{complex!projective space!geodesics}

The inverse of the Fubini--Study metric, as written in
equation (\ref{e:fsmetric}), is easily shown to be
\begin{equation}
g^{\overline{i}j} = \frac{1}{4}\left( 
                    1+\left| z\right|^2 
                \right) \left[
                    \delta^{ij} + \overline{z}^iz^j
                \right].
\label{e:fsinvmetric}
\end{equation}
The only non--vanishing Christoffel symbols are $\Gamma^a_{bc}$ and
$\Gamma^{\overline{a}}_{\overline{b}\overline{c}}$ ( $=\overline{\Gamma^a_{bc}}$ ), 
\index{complex!projective space!Christoffel symbols}
so the geodesic equations become
\begin{equation}
\begin{split}
    \ddot{z}^l + \Gamma^l_{ik}\dot{z}^i\dot{z}^k =&\, 0\\
    \ddot{\overline{z}}^l + \Gamma^{\overline{l}}_{\overline{i}\overline{k}}
            \dot{\overline{z}}^i\dot{\overline{z}}^k =&\, 0.
\end{split}
\label{e:geodesics}
\end{equation}
\index{complex!projective space!geodesics}
Metric compatibility ($\nabla_kg_{i\overline{j}} = \nabla_{\overline{k}}
g_{i\overline{j}} = 0$) insures
\begin{equation}
\begin{split}
    \Gamma^a_{bc} =& g^{\overline{s}a} g_{c\overline{s},b}\\
    \Gamma^{\overline{a}}_{\overline{b}\overline{c}} =& g^{\overline{a}s} 
                            g_{s\overline{c},b}.
\end{split}
\label{e:xsymb}
\end{equation}
Using equation (\ref{e:fsmetric}), we can calculate partials of the metric 
\begin{equation}
\begin{split}
    g_{i\overline{j},k} =&\,4\, \frac{\partial}{\partial z^k} 
            \left(
                 \frac{\delta_{ij}\left( 1 + \left| z \right|^2 \right) - z_i\overline{z}_j }
                      { \left( 1 + \left| z\right|^2 \right)^2 }
            \right)\\
        =&\,4\, \left\lbrace
            \frac{\partial}{\partial z^k} \left(
                \frac{\delta_{ij}}{ \left( 1 + \left| z\right|^2 \right) }
            \right)
            - \frac{\partial}{\partial z^k} \left(
                \frac{z_i\overline{z}_j}{ 
                    \left( 1 + \left| z\right|^2 \right)^2 
                }
            \right)
        \right\rbrace\\
        =&\,4\,\left\lbrace
            -\frac{\delta_{ij}\overline{z}_k}{ 
                \left( 1 + \left| z\right|^2 \right)^2 
             }
            +2\,\frac{z_i\overline{z}_j\overline{z}_k}{
                    \left( 1 + \left| z\right|^2 \right)^3 
             }
            -\frac{\delta_{ik}\overline{z}_j}{ 
                \left( 1 + \left| z\right|^2 \right)^2 
             }
        \right\rbrace\\
        =&\,\frac{4}{
                \left( 1 + \left| z\right|^2 \right)^3 
            }
            \biggl\lbrace
                2\,z_i\overline{z}_j\overline{z}_k
                - \left( 1 + \left| z\right|^2 \right) \left[
                    \delta_{ij}\overline{z}_k 
                    + \delta_{ik}\overline{z}_j 
                \right]
            \biggr\rbrace,
\end{split}
\end{equation}
and
\begin{equation}
    g_{i\overline{j},\overline{k}} 
        =\,\frac{4}{
                \left( 1 + \left| z\right|^2 \right)^3 
            }
            \biggl\lbrace
                2\,z_i\overline{z}_jz_k
                - \left( 1 + \left| z\right|^2 \right) \left[
                    \delta_{ij}z_k 
                    + \delta_{jk}z_i 
                \right]
            \biggr\rbrace.
\end{equation}
Equation (\ref{e:xsymb}) gives Christoffel symbols
\begin{equation}
\begin{split}
    \Gamma^l_{ik} =& g^{\overline{j}l} g_{i\overline{j},k}\\
        =& \frac{1}{ \left( 1 + \left| z\right|^2 \right)^2 }
        \biggl\lbrace
            \delta^{jl} + \overline{z}^jz^l
        \biggr\rbrace
        \biggl\lbrace
            2z_i\overline{z}_j\overline{z}_k
            - \left( 1 + \left| z\right|^2 \right) \left[
                \delta_{ij}\overline{z}_k 
                + \delta_{ik}\overline{z}_j 
            \right]
        \biggr\rbrace\\
        =& 
        \frac{1}{ \left( 1 + \left| z\right|^2 \right)^2 }
        \biggl\lbrace
            2z_i\overline{z}^l\overline{z}_k
            - \left( 1 + \left| z\right|^2 \right) \left[
                \delta_i^l\overline{z}_k 
                + \delta_{ik}\overline{z}^l 
            \right]\\
        &\qquad\qquad\qquad    + 2\left( \overline{z}\cdot 
                                         \overline{z}\right) z^lz_i\overline{z}_k
            - \left( 1 + \left| z\right|^2 \right) \left[
                \overline{z}_i z^l\overline{z}_k 
                + \left( \overline{z}\cdot \overline{z}\right)\delta_{ik}z^l 
            \right]
        \biggr\rbrace,
\end{split}
\end{equation}
and
\begin{multline}
    \Gamma^{\overline{l}}_{\overline{i}\overline{k}} 
        = 
        \frac{1}{ \left( 1 + \left| z\right|^2 \right)^2 }
        \biggl\lbrace
            2\overline{z}_iz^lz_k
            - \left( 1 + \left| z\right|^2 \right) \left[
                \delta_i^lz_k 
                + \delta_{ik}z^l 
            \right]\\
        + 2\left( z\cdot z\right) \overline{z}^l\overline{z}_iz_k
            - \left( 1 + \left| z\right|^2 \right) \left[
                z_i \overline{z}^lz_k 
                + \left( z\cdot z\right)\delta_{ik}\overline{z}^l 
            \right]
        \biggr\rbrace.
\end{multline}
The geodesic equations of motion (\ref{e:geodesics}) now become
$\dot{z}=w$,
\begin{multline}
    \dot{w}^l = \frac{1}{ \left( 1 + \left| z\right|^2 \right)^2 }
        \biggl\lbrace
            \biggl[ \left( 1 + \left| z\right|^2 \right) 
                    \left( \overline{z}\cdot w \right)
            \biggr] w^l\\
            +
            \biggl[ \left( 1 + \left| z\right|^2 \right)\left( w\cdot w \right)
                    - 2\left( z\cdot w \right)\left( \overline{z}\cdot w \right)
            \biggr] \overline{z}^l\\
            +
            \biggl[ \left( 1 + \left| z\right|^2 \right) 
                    \bigl[ 
                      \left(\overline{z}\cdot w\right)\left(\overline{z}\cdot w\right)
                      +\left(\overline{z}\cdot\overline{z}\right)\left(w\cdot w\right)
                    \bigr]
                    - 2\left(\overline{z}\cdot\overline{z}\right) 
                        \left( z\cdot w \right)\left( \overline{z}\cdot w \right)
            \biggr] z^l
        \biggr\rbrace,
\end{multline}
along with Hermitian conjugates $\dot{\overline{z}}=\overline{w}$ and 
\begin{multline}
    \dot{\overline{w}}^l = \frac{1}{ \left( 1 + \left| z\right|^2 \right)^2 }
        \biggl\lbrace
            \biggl[ \left( 1 + \left| z\right|^2 \right) 
                    \left( z\cdot \overline{w} \right)
            \biggr] \overline{w}^l\\
            +
            \biggl[ \left( 1 + \left| z\right|^2 \right)\left( 
                            \overline{w}\cdot \overline{w} \right)
                    - 2\left( \overline{z}\cdot \overline{w} \right)
                        \left( z\cdot \overline{w} \right)
            \biggr] z^l\\
            +
            \biggl[ \left( 1 + \left| z\right|^2 \right) 
                    \bigl[ 
                      \left(z\cdot \overline{w}\right)\left(z\cdot \overline{w}\right)
                      +\left(z\cdot z\right)\left(\overline{w}\cdot\overline{w}\right)
                    \bigr]
                    - 2\left(z\cdot z\right) 
                        \left( \overline{z}\cdot \overline{w} \right)
                        \left( z\cdot \overline{w} \right)
            \biggr] \overline{z}^l
        \biggr\rbrace.
\end{multline}


%--------------------------------------------------------------------------------   
%--------------------------------------------------------------------------------   
\section{Mixed States}
\index{mixed states}

Based on the work of Bures\cite{Bures:69},
Uhlmann\cite{Uhlmann:76} obtained an explicit expression for the 
distance between two mixed--state density matrices,
$\rho_1$ and $\rho_2$.  This 
distance is given by
\begin{equation}
d_{\text{Bures}}\left(\rho_1,\rho_2\right)
= \sqrt{2} \left\lbrace
    1 - \tr\left[ 
            \left( \rho_1^\frac{1}{2} \rho_2 \rho_1^\frac{1}{2}
            \right)^\frac{1}{2}
           \right]
\right\rbrace^\frac{1}{2}.
\end{equation}
Note that, for two neighboring density matrices, 
$\rho$ and $\rho + d\rho$,
this distance coincides with the Riemannian metric on the
space of density matrices (not just pure) obtained by
Braunstein and Caves\cite{Braunstein/Caves:94}.
The finite distance described above is sufficient 
for purposes of this dissertation.
