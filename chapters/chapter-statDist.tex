
%---------------------------------------------------------------------------------------
%---------------------------------------------------------------------------------------
\chapter{Statistical distance and Hilbert space}
\label{sec:statisticalDistance}
\index{Statistical Distance@\emph{Statistical Distance}}%


In classical probability, the \emph{distance} between two distinguishable 
events in state space can be characterized in terms of the maximum number
of mutually distinguishable intermediate events.  This notion was 
generalized to quantum states by Wooters\cite{Wooters:81}, who found
that the \emph{statistical distance} between two quantum states coincides with
the \emph{metric distance} between two rays in Hilbert space.

Optimization of the statistical distinguishability of quantum
states leads to a natural Hermitian metric on the space of density
operators\cite{Braunstein/Caves:94}.  This metric is, for pure states,
the usual Fubini--Study metric on projective Hilbert space\cite{Gibbons:92}.
It can be extended to general density operators with little difficulty.

\section{Classical statistical distance}

\mmm{Is this really necessary?}

It is worth noting, that statistical distance on a classical state space 
is not just the Euclidean distance between two probabilities $|p_1-p_2|$.  
This is due to the fact that probabilities near $\frac{1}{2}$ are harder 
to distinguish due to higher statistical fluctuations than probabilities 
near 0 or 1.

As an example, consider two differently weighted coins.  Coin $i$ has
an {\it a priori} probability $p_i$ of obtaining heads in a given toss.
The probability space is one dimensional, just the probability 
$p_i$ of heads for each coin.
Define\footnote{After Wooters\cite{Wooters:81}} 
the statistical distance between two coins in this probability space as
\begin{equation}
d(p_1,p_2) = \lim_{n\to\infty} \frac{1}{\sqrt{n}}\times
    \left[\begin{matrix}
        \text{\scriptsize maximum number of mutually distinguishable}\\
        \text{\scriptsize (in {\it n} trials) intermediate probabilities}
    \end{matrix}\right],
\label{e:ClassicalDistance}
\end{equation}
where two intermediate probabilities $p$ and $p^\prime$ are distinguishable
in $n$ trials if 
\begin{equation}
\left| p-p^\prime \right| \ge \Delta p + \Delta p^\prime.
\end{equation}
An expression for $\Delta p$ can be given in the spirit of experimental
uncertainty as RMS deviation
\begin{equation}
\Delta p = \left[ \frac{p(1-p)}{n} \right]^\frac{1}{2}.
\end{equation}
\mmm{???}


\section{A geometrical approach}

I'd like to characterize ``Statistical Distance'' as defined by Wooters\cite{Wooters:81}
geometrically.  I'll follow Braunstein and Caves\cite{Braunstein/Caves:94}
to get the Fubini--Study metric for projective Hilbert space\dots
\begin{equation}
                ds_{\text{SP}}^2 = \sum_i \frac{(dp_i)^2}{p_i} + 
                \left( 
                    \sum_i p_i(d\phi_i)^2 - \left( \sum_i p_i d\phi_i \right)^2
                \right)     
\end{equation}

Consider two pure states:
\begin{equation}
\begin{split}
    \ket{\psi} =& \sum_j \sqrt{p_j}e^{i\phi_j}\ket{j},\\
    \ket{\tilde\psi} = \ket{\psi} + \ket{d\psi} =& 
              \sum_j \sqrt{p_j + dp_j}e^{i(\phi_j + d\phi_j)}\ket{j}.
\end{split}
\end{equation}
If we define $\ket{d\psi_\perp} = \ket{d\psi} - \ket{\psi}\braket{\psi}{d\psi}$, 
then
\begin{equation}
    \braket{d\psi_\perp}{d\psi_\perp} = 
        \braket{d\psi}{d\psi} - \left| \braket{\psi}{d\psi} \right|^2.
\label{e:dPsi_perp}
\end{equation}
Since $\ket{d\psi} = \ket{\tilde\psi} - \ket{\psi}$, we can write
\begin{equation}
\begin{split}
    \braket{d\psi}{d\psi} 
        =& \left(\bra{\tilde\psi}-\bra{\psi}\right)
           \left(\ket{\tilde\psi}-\ket{\psi}\right)\\
        =& \braket{\tilde\psi}{\tilde\psi} - \braket{\tilde\psi}{\psi}
          - \braket{\psi}{\tilde\psi} + \braket{\psi}{\psi}\\
        =& \,2 - \braket{\psi}{\tilde\psi} - \braket{\tilde\psi}{\psi}\\
        =& \,2 - \left(
                      \sum_j\sqrt{p_j}\sqrt{p_j+dp_j}e^{id\phi_j}
                \right) 
                - \left(
                      \sum_j\sqrt{p_j}\sqrt{p_j+dp_j}e^{-id\phi_j}
                \right)\\
        =& \,2 - 2 \sum_j\sqrt{p_j}\sqrt{p_j+dp_j}
                \,\cos(d\phi_j).
\end{split}
\end{equation}
Now, we can expand to second order in $d\phi$ and $dp$ to get
\begin{equation}
\begin{split}
    \braket{d\psi}{d\psi}
        \approx& \,2\left\lbrace1 - \sum_jp_j\left(1+\frac{dp_j}{p_j}\right)^\frac{1}{2} 
                    \left[ 1 - \frac{(d\phi_j)^2}{2}\right]
            \right\rbrace\\
        \approx& \,2\left\lbrace1 - \sum_jp_j
                    \left[1+\frac{1}{2}\left(\frac{dp_j}{p_j}\right)
                           -\frac{1}{8}\left(\frac{dp_j}{p_j}\right)^2\right]
                    \left[ 1 - \frac{(d\phi_j)^2}{2}\right]
            \right\rbrace\\
        =& \,2\left\lbrace1 - \sum_j
                    \left( p_j + \frac{1}{2}dp_j 
                           -\frac{1}{8}\frac{(dp_j)^2}{p_j}
                           -\frac{1}{2}p_j(d\phi_j)^2
                    \right)
            \right\rbrace.\\
\end{split}
\end{equation}
Now, $\lbrace p_i\rbrace$ are probabilities, so $\sum_jp_j=1$ and $\sum_jdp_j = 0$,
which gives
\begin{equation}
    \braket{d\psi}{d\psi}
        \approx \sum_j 
                    \left( 
                           \frac{1}{4}\frac{(dp_j)^2}{p_j}
                           +p_j(d\phi_j)^2
                    \right).
\end{equation}
%
Similarly, 
\begin{equation}
\begin{split}
    \braket{\psi}{d\psi}
        =& \bra{\psi} \left(\ket{\tilde\psi} - \ket{\psi}\right)
        = \braket{\psi}{\tilde\psi} - \braket{\psi}{\psi}\\
        =& \braket{\psi}{\tilde\psi} - 1\\
        =& \sum_j\sqrt{p_j}\sqrt{p_j+dp_j}\,e^{id\phi_j} - 1
\end{split}
\end{equation}
Expanding to first order in $dp$ and $d\phi$, we get
\begin{equation}
\begin{split}
    \braket{\psi}{d\psi}
        \approx& \left\lbrace \sum_jp_j
                    \left[1+\frac{1}{2}\left(\frac{dp_j}{p_j}\right)
                           %-\frac{1}{8}\left(\frac{dp_j}{p_j}\right)^2
                    \right]
                    \left[ 1 + id\phi_j 
                            %- \frac{(d\phi_j)^2}{2}
                    \right]
                 \right\rbrace
            - 1\\ 
        \approx& \sum_j \left( p_j + \frac{1}{2}dp_j + ip_jd\phi_j
                  \right) -1\\
        =& i \sum_j p_jd\phi_j.
\end{split}
\end{equation}
and then equation (\ref{e:dPsi_perp}) gives
\begin{equation}
\begin{split}
    \braket{d\psi_\perp}{d\psi_\perp} 
        =& \braket{d\psi}{d\psi} - \left| \braket{\psi}{d\psi} \right|^2.\\
        \approx& \sum_j 
                    \left( 
                           \frac{1}{4}\frac{(dp_j)^2}{p_j}
                           +p_j(d\phi_j)^2
                    \right)
            - \left( \sum_jp_jd\phi_j \right)^2\\
        \approx& \sum_j \frac{1}{4}\frac{(dp_j)^2}{p_j}
                  + \left[ 
                           \sum_j p_j(d\phi_j)^2
                           - \left( \sum_jp_jd\phi_j \right)^2
                    \right].
\end{split}
\end{equation}


%\lemma This sucks. This really sucks. to the $\tr\rho$, or
%maybe $\tr{\rho}$, or maybe even $\tr[\rho]$.  Is this $\C$, or
%does $\ket{i}$.
%
%\proof This is the proof.\qed
