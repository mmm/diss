% chapter-code.tex

%\baselineskip=15.5pt plus .5pt minus .2pt

\chapter{The Code}
\label{chap:code}
\index{Code@\emph{Code}}

\baselineskip=15.5pt plus .5pt minus .2pt

This chapter contains details of the computer program used
in the numerical simulation of Grover's algorithm.
The code itself is interesting and possibly useful in
other quantum simulations due to the general nature of the
representations of quantum states.

This code is written in C++ and compiled with development
snapshots of the GNU C++ compiler (\texttt{http://gcc.gnu.org/}).
Any compiler compliant with the ANSI ISO C++ standard \emph{should}
compile the code, but I would read about how the compiler implements
and instantiates templates before I tried it.


%-----------------------------------------------------------------
%-----------------------------------------------------------------
\section{Classes Used}

%-----------------------------------------------------------------
\subsection{State}
\lgrindfile{figures/State.h.lg}
%-----------------------------------------------------------------
\subsection{PureState}
\lgrindfile{figures/PureState.h.lg}
%-----------------------------------------------------------------
\subsection{MixedState}
\label{ssec:mixedstate}
\lgrindfile{figures/MixedState.h.lg}


%-----------------------------------------------------------------
%-----------------------------------------------------------------
\section{Libraries Used}

%-----------------------------------------------------------------
\subsection{The \texttt{valarray} Template}
\index{Code@\emph{Code}!Valarray}

%-----------------------------------------------------------------
\subsection{The Template Numerical Toolkit}

%-----------------------------------------------------------------
\subsection{blitz++}

%-----------------------------------------------------------------
\subsection{LAPACK}
The module \texttt{fwrap.C} is an example o


%-----------------------------------------------------------------
%-----------------------------------------------------------------
\section{Modules}

\begin{sloppy}

%-----------------------------------------------------------------
\subsection{main}
\lgrindfile{figures/main.C.lg}
%-----------------------------------------------------------------
%\subsection{State}
%\lgrindfile{figures/State.h.lg}
%-----------------------------------------------------------------
\subsection{PureState}
%\lgrindfile{figures/PureState.h.lg}
\lgrindfile{figures/PureState.C.lg}
%-----------------------------------------------------------------
\subsection{MixedState}
%\lgrindfile{figures/MixedState.h.lg}
\lgrindfile{figures/MixedState.C.lg}
%-----------------------------------------------------------------
\subsection{derivs}
\lgrindfile{figures/derivs.C.lg}
%-----------------------------------------------------------------
\subsection{rk4}
\lgrindfile{figures/rk4.C.lg}
%-----------------------------------------------------------------
\subsection{Bures}
\lgrindfile{figures/Bures.C.lg}
%-----------------------------------------------------------------
\subsection{Matrices}
\lgrindfile{figures/Matrices.h.lg}
\lgrindfile{figures/Matrices.C.lg}
%-----------------------------------------------------------------
\subsection{display}
\lgrindfile{figures/display.C.lg}
%-----------------------------------------------------------------
\subsection{exceptions}
\lgrindfile{figures/exceptions.h.lg}
\lgrindfile{figures/exceptions.C.lg}
%-----------------------------------------------------------------
\subsection{fwrap}
\lgrindfile{figures/fwrap.C.lg}
%-----------------------------------------------------------------
\subsection{myvalarray}
\lgrindfile{figures/myvalarray.h.lg}

\end{sloppy}
