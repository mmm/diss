% chapter-code.tex

%\baselineskip=15.5pt plus .5pt minus .2pt

\chapter{The Code}
\label{chap:code}
\index{Code@\emph{Code}}

\baselineskip=15.5pt plus .5pt minus .2pt

This chapter contains details of the computer program used
in the numerical simulation of Grover's algorithm.
The code itself is interesting and possibly useful in
other quantum simulations due to the general nature of the
representations of quantum states.

This code is written in C++ and compiled with development
snapshots of the GNU C++ compiler (\texttt{http://gcc.gnu.org/}).
Any compiler compliant with the ANSI ISO C++ standard \emph{should}
compile the code, but I would read about how the compiler implements
and instantiates templates before I tried it.  The makeline
used included
\begin{verbatim}
CFLAGS = -ansi -pedantic -Wall -O -gstabs+ -D_REENTRANT 
CFLAGS += -D_GNU_SOURCE -ftemplate-depth-17 -funroll-loops 
CFLAGS += -fstrict-aliasing
LIBS = -llapack -lblas -lg2c -lm
\end{verbatim}



%-----------------------------------------------------------------
%-----------------------------------------------------------------
\section{Classes Used}

%-----------------------------------------------------------------
\subsection{A General Quantum State}
\lgrindfile{figures/State.h.lg}
%-----------------------------------------------------------------
\subsection{A Pure Quantum State}
\lgrindfile{figures/PureState.h.lg}
%-----------------------------------------------------------------
\subsection{A Mixed Quantum State}
\label{ssec:mixedstate}
\lgrindfile{figures/MixedState.h.lg}


%-----------------------------------------------------------------
%-----------------------------------------------------------------
\section{Libraries Used}

\begin{itemize}

\item
The ANSI/ISO standard C++ library includes template support,
which is used heavily throughout this program.
In particular, the \texttt{valarray} 
\index{Code@\emph{Code}!Valarray}
template combines the ease of using a template--based
container with the numerical speed of a simple
C array.  The implementation for templates within the g++
compiler suite is still under development, so a heavy
regimen of testing is recommended, paricularly with
templates for expression evaluation.
For more information about 
templates in C++ see\cite{Stroustrup:97}, and
for more information on how these are implemented
by the GNU compilers check out
\texttt{http://gcc.gnu.org/}.

\item
The Template Numerical Toolkit \texttt{TNT} 
Provides fairly functional matrix interfaces
and compiles quickly despite being entirely
template based.  More on this alleged successor to
things like \texttt{lapack++} can be found at
\texttt{http://math.nist.gov/tnt}.


\item
The \texttt{blitz++} toolkit seems like it will be a great toolkit
when completed.  I simply used the component of the library
that generates random numbers, preferring \texttt{TNT} 
for matrix work.  This was done for two reasons,
the blitz library is all templates and took \emph{forever}
to compile even the simplest things under the GNU compiler,
and \texttt{TNT} provided examples for interfacing 
with \texttt{LAPACK} routines, where \texttt{blitx++}
did not.
More information about this library can be found at
\texttt{http://www.oonumerics.org}.

\item
The ubiquitous Fortran library \texttt{LAPACK} was used 
for the rather droll
task of diagonalizing density matrices.  One of my primary
reasons for including code in this dissertation is to provide
examples of calling these legacy library routines from C++.
The module \texttt{fwrap.C} is example code from the 
\texttt{TNT}
slightly adapted to do eigenvalue and eigenvector problems
for double precision complex Hermitian matrices, and is
included below.

\end{itemize}

%-----------------------------------------------------------------
%-----------------------------------------------------------------
\section{Modules}


%-----------------------------------------------------------------
\subsection{main}
\lgrindfile{figures/main.C.lg}
%-----------------------------------------------------------------
%\subsection{State}
%\lgrindfile{figures/State.h.lg}
%-----------------------------------------------------------------
\subsection{PureState}
%\lgrindfile{figures/PureState.h.lg}
\lgrindfile{figures/PureState.C.lg}
%-----------------------------------------------------------------
\subsection{MixedState}
%\lgrindfile{figures/MixedState.h.lg}
\lgrindfile{figures/MixedState.C.lg}
%-----------------------------------------------------------------
\subsection{derivs}
\lgrindfile{figures/derivs.C.lg}
%-----------------------------------------------------------------
\subsection{rk4}
\lgrindfile{figures/rk4.C.lg}
%-----------------------------------------------------------------
\subsection{Bures}
\lgrindfile{figures/Bures.C.lg}
%-----------------------------------------------------------------
\subsection{Matrices}
\lgrindfile{figures/Matrices.h.lg}
\lgrindfile{figures/Matrices.C.lg}
%-----------------------------------------------------------------
\subsection{display}
\lgrindfile{figures/display.C.lg}
%-----------------------------------------------------------------
\subsection{exceptions}
\lgrindfile{figures/exceptions.h.lg}
\lgrindfile{figures/exceptions.C.lg}
%-----------------------------------------------------------------
\subsection{fwrap}
\lgrindfile{figures/fwrap.C.lg}
%-----------------------------------------------------------------
\subsection{myvalarray}
\lgrindfile{figures/myvalarray.h.lg}
