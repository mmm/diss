\documentclass{slides}
\usepackage{times}
\usepackage{epsfig}
\usepackage{amsmath,amsthm,amsfonts,amscd} 
\usepackage{epic}       % This is to draw pictures
\usepackage{eepic}       % This is to draw pictures


\newcommand{\mmm}[1]{$\bullet\bullet\bullet$\,{\tiny\emph{{#1}}}\,$\bullet\bullet\bullet$}
\newcommand{\ie}{{\it i.e.}}
\newcommand{\cf}{{\it c.f.}}



%\newtheorem{defi}{Definition}[chapter]

\newtheorem{lemma}{Lemma}
%\newtheorem{lemma}[defi]{Lemma}

%\newtheorem{satz}[defi]{Theorem}
%\newtheorem{cor}[defi]{Corollary}
%\newtheorem{bem}[defi]{Remark}
%\newtheorem{prop}[defi]{Proposition}
%\newtheorem{exempel}[defi]{Example}
%\newtheorem{quest}[defi]{Question}
%\newtheorem{conj}[defi]{Conjecture}



%\newcommand{\proof}{{\emph{Proof.\ }}}
%\newcommand{\qed}{\hfill $\Box$}
\newcommand{\tr}{{\operatorname{Tr}\,}}
\newcommand{\id}{{\operatorname{id}}}
\newcommand{\supp}{{\operatorname{supp}\,}}
\newcommand{\bra}[1]{{\langle{#1}|}}
\newcommand{\ket}[1]{{|{#1}\rangle}}
\newcommand{\braket}[2]{{\langle{#1}|{#2}\rangle}}
\newcommand{\ketbra}[2]{{|{#1}\rangle\!\langle{#2}|}}
\newcommand{\C}{{\mathbb{C}}}
\newcommand{\CP}{{\mathbb{C} P}}
\newcommand{\R}{{\mathbb{R}}}
\newcommand{\N}{{\mathbb{N}}}
\newcommand{\Z}{{\mathbb{Z}}}
\newcommand{\alg}[1]{{\mathfrak{#1}}}
\newcommand{\fset}[1]{{\mathcal{#1}}}
\newcommand{\Rho}{{\sf P}}
\newcommand{\Sym}{{\operatorname{Sym}}}
\newcommand{\1}{{\mathbbm{1}}}
\newcommand{\im}{{\operatorname{im}}}
\newcommand{\lcsupp}{{\operatorname{l.c.supp}}}
\newcommand{\Span}{{\operatorname{span}}}
%
% folgende beispiel-environments: einmal mit, einmal ohne namen
%\newenvironment{expl}[1][{}]{\begin{exempel}[#1]\normalfont}{\end{exempel}}
%\newenvironment{expl_noname}{\begin{exempel}\normalfont}{\end{exempel}}
%


\begin{document}

\begin{center}
\textbf{Noise in Grover's Quantum Search Algorithm}\\
\bigskip
\bigskip
{An Overview}
\end{center}

\begin{itemize}
\item Why Quantum Computers?
\item Shor's Algorithm.
\item Grover's Algorithm.
\item Adding Noise.
\item Results.
\end{itemize}

\pagebreak

%-----------------------------------------------------------------

\begin{center}
\textbf{Why Quantum Computers?}
\end{center}

The function $f\colon\Z\to\Z$,
when represented on a classical computer, where
integers are represented by
\begin{equation*}
x=(\underbrace{0110\ldots0}_{n\text{ bits}}),% \in\Z_{2^n},
\end{equation*}
becomes $f\colon\Z_{2^n}\to\Z_{2^n}$
\begin{equation*}
x\quad\mapsto\quad f(x).
\end{equation*}
\begin{itemize}
\item Is this function constant?
\item Is this function monotonic?
\item Is this function periodic?  If so, find the period.
\end{itemize}
Questions such as these are \emph{very} difficult to answer.

\begin{center}
{\sl Can we do better?}
\end{center}

\pagebreak

Represent an integer $x$ by the state of some quantum system
\begin{equation*}
\begin{split}
\ket{x}=&\ket{\underbrace{0110\ldots0}_{n\text{ qubits}}}\\
 =& \ket{0}\otimes
             \ket{1}\otimes\ldots,
%             \ket{1}\otimes
%             \ket{0}\otimes\ldots,
\end{split}
\end{equation*}
where
\begin{equation*}
\begin{split}
    \ket{1} =& \,\ket{\uparrow}\\
    \ket{0} =& \,\ket{\downarrow}.
\end{split}
\end{equation*}

Now, evaluating a function $f$ should consist of some physical
process taking the state $\ket{x}$ to some
other state $\ket{f(x)}$.  This can be given by a unitary
operator
\begin{equation*}
{\bf U_f}\ket{x}\ket{0} = \ket{x}\ket{f(x)}.
\end{equation*}

\begin{center}
Big Deal.  

{\sl Does this really change anything?}
\end{center}

\pagebreak

\begin{center}
{\bf Yes!}
\end{center}

%A gate in a classical computer can only operate on a bit in a 
%state of either 1 or 0.
%
%A gate in a quantum computer can also act on a qubit in a state 
%of $\ket{1}$ or $\ket{0}$.  
%However, a quantum gate can also act on a qubit
%whose state is some superposition $a\ket{0}+b\ket{1}$ of $\ket{0}$
%and $\ket{1}$.

A \emph{quantum} gate can act on a state in a superposition
%$a\ket{0}+b\ket{1}$ 
\begin{equation*}
\frac{1}{\sqrt{|a|^2+|b|^2}}\left(a\ket{0}+b\ket{1}\right)
\end{equation*}
of $\ket{0}$
and $\ket{1}$.

Consider the equally--weighted superposition of all possible 
integers $x\in \Z_{2^n}$
\begin{equation*}
\frac{1}{\sqrt{2^n}}\sum_{x=0}^{2^n-1} \ket{x}.
\end{equation*}

Now, evaluate the function $f$
\begin{equation*}
{\bf U_f} \left\lbrack
\frac{1}{\sqrt{2^n}}\sum_{x=0}^{2^n-1}\ket{x}\ket{0}
\right\rbrack
= 
\frac{1}{\sqrt{2^n}}\sum_{x=0}^{2^n-1}\ket{x}\ket{f(x)}.
\end{equation*}

All information about the function is obtained after a single
function evaluation!

\pagebreak

\begin{center}
\textbf{Finding the period of a function}
\end{center}

%Given some periodic function $f$, evaluate the function 
Given some periodic function $f$:
\begin{enumerate}
\item Evaluate the function 
(as before) on an equally weighted superposition of all 
inputs 
\begin{equation*}
{\bf U_f} \left\lbrack
\frac{1}{\sqrt{2^n}}\sum_{x=0}^{2^n-1}\ket{x}\ket{0}
\right\rbrack
= 
\frac{1}{\sqrt{2^n}}\sum_{x=0}^{2^n-1}\ket{x}\ket{f(x)}.
\end{equation*}
\item Extract information about the period by performing
a measurement
\begin{equation*}
\bigl(
{\bf 1}\otimes\ketbra{c}{c}
\bigr)
\sum_{x=0}^{2^n-1}\ket{x}\ket{f(x)}
= \sum_{\lbrace x|f(x)=c\rbrace} \ket{x}\ket{c},
\end{equation*}
\item Obtain the period by taking a Fourier
transform of the first register.
\end{enumerate}


\pagebreak

%-----------------------------------------------------------------

\begin{center}
\textbf{Shor's Algorithm}\\
\bigskip
Factoring Large Integers
\end{center}

\pagebreak

\begin{center}
\textbf{Shor's Algorithm}\\
\bigskip
An Example
\end{center}

%Consider some integer $N=pq>>1$, where $p$ and $q$ are prime.
Take the number $N=91$ and the function
\begin{equation*}
f(x) = 3^x\pmod{91}.
\end{equation*}
\begin{flushleft}
\begin{tabular}{rrrrrrrrl}
x:& 0,& 1,& 2,& 3,& 4,& 5,& 6,& \dots\\
$3^x$:& 1,& 3,& 9,& 27,& 81,& 243,& 729,& \dots\\
%$3^x\pmod{91}$:& 1,& 3,& 9,& 27,& 81,& 61,& 1,& \dots
$f(x)$:& 1,& 3,& 9,& 27,& 81,& 61,& 1,& \dots
\end{tabular}
\end{flushleft}
By inspection, the period of $f$ is 6, and so 
\begin{equation*}
\begin{split}
3^6- 1\cong&\ 0 \pmod{91}\\
\left(3^3+1\right)
\left(3^3-1\right)
\cong&\ 0 \pmod{91}\\
(28)(26)\cong&\ 0\pmod{91}.
\end{split}
\end{equation*}
This implies that either
\begin{equation*}
\gcd\left(91,26\right) = 13
\end{equation*}
or
\begin{equation*}
\gcd\left(91,28\right) = 7
\end{equation*}
will be a non--trivial factor of 91.


\pagebreak

%-----------------------------------------------------------------

\begin{center}
\textbf{Grover's Algorithm}\\
\bigskip
Searching a Quantum Database
\end{center}

\pagebreak

\begin{center}
\textbf{Grover's Algorithm}\\
\bigskip
A \emph{Quantum} Database
\end{center}

\begin{equation*}
\left\lbrace
    \begin{matrix}
    \text{$2^n$ items in}\\
    \text{database}
    \end{matrix}
\right\rbrace
\qquad\mapsto\qquad
\left\lbrace
    \begin{matrix}
    \text{$2^n$ quantum}\\
    \text{states}\\
    \text{($n$ qubits)}
    \end{matrix}
\right\rbrace
\end{equation*}

Grover's search algorithm starts with the database in an
equally weighted superposition of all states
\begin{equation*}
\ket{\Psi} = \sum_{i} a_i\ket{i}
\qquad \left(a_i = \frac{1}{\sqrt{N}}\text{ for all }i\right)
\end{equation*}
and then gradually amplifies the coefficient of the desired state
$\ket\omega$
while simultaneously suppressing the coefficients of all others
$\ket{\omega_\perp}$
\begin{equation*}
\begin{split}
%a_{\text{desired}}&\mapsto 1\\
%a_{\left\lbrace\text{all others}\right\rbrace}&\mapsto 0.
a_{\omega}&\mapsto 1\\
a_{\left\lbrace\omega_\perp\right\rbrace}&\mapsto 0.
\end{split}
\end{equation*}

Successful completion of the algorithm leaves $\ket\Psi$ in the
desired state $\ket\omega$ with high probability.

\pagebreak
%
%To see how Grover's algorithm works, let
%\begin{equation*}
%\begin{split}
%\ket{\text{Desired State}} &= \ket\omega\\
%\ket{\text{all others}} &= \ket{\omega_\perp}.
%\end{split}
%\end{equation*}
%
%The state of the database at any time can be written
%\begin{equation*}
%\ket\Psi = a_\omega \ket\omega + 
%\sum_{\omega_\perp} a_{\omega_\perp}\ket{\omega_\perp}.
%\end{equation*}
%
%\pagebreak

Grover's algorithm consists of successive iterations of the following
process:
\begin{enumerate}
\item Query the Oracle. \ie, flip the phase of the desired state
\begin{equation*}
{\bf U_\omega}\ket\Psi = \biggl( {\bf 1} - 2\ketbra{\omega}{\omega} \biggr) \ket{\Psi}
\end{equation*}
\begin{center}
\begin{picture}(150,150)
    \thicklines
    \path(0,0)(20,50)
    \path(20,50)(150,70)
    \path(150,70)(130,20)
    \path(130,20)(0,0)
    \put(75,35){\vector(-1,4){20}}
    \put(20,20){$\ket{\omega_\perp}$}
    \put(65,100){$\ket{\omega}$}
    \put(140,90){$\ket{\Psi}$}
    \put(145,-5){${\bf U_\omega}\ket{\Psi}$}
    \put(75,35){\vector(1,1){60}}
    \dashline[+30]{3}[80](75,35)(102,16)
    \put(102,16){\vector(3,-2){40}}
\end{picture}
\end{center}
\item and then invert the resulting state about the ``average'' state
$\ket{s} = \frac{1}{\sqrt{N}}\sum_{i=0}^{N-1} \ket{i}$
\begin{equation*}
%{\bf U_s} = 2\ketbra{s}{s} - {\bf 1}.
{\bf U_s}\ket\Psi = \biggl( 
2\ketbra{s}{s} - {\bf 1}
\biggr) \ket{\Psi}
\end{equation*}
\end{enumerate}






\pagebreak

%-----------------------------------------------------------------

\begin{center}
\textbf{Adding Noise}\\
\bigskip
A dynamical treatment of Grover's algorithm
\end{center}

\pagebreak

Grover's algorithm can be described by a geodesic on the 
space of pure quantum states.
\begin{center}
\begin{picture}(200,150)
    \thicklines
    \qbezier(0,70)(40,150)(120,100)
    \qbezier(120,100)(150,80)(180,85)
    \qbezier(0,70)(-40,-50)(140,0)
    \qbezier(180,85)(270,90)(200,30)
    \qbezier(200,30)(170,10)(140,0)
    \put(20,20){$\ket{\Psi}$}
    \put(110,70){$\ket{\omega}$}
    \qbezier[15](25,35)(70,70)(175,60)
\end{picture}
\end{center}

If a numerical model can integrate the geodesic equations of
motion, noise can then be added in at every step.

\vfil
\begin{center}
{\sl This is not good enough.}
\end{center}

\pagebreak

The numerical model must be modified to include 
mixed states of the form
\begin{equation*}
\rho = \sum\lambda_i\rho_i.
\end{equation*}

Noise must then be added to both the pure state 
evolution (the $\rho_i$'s) and the mixture of 
the state (the $\lambda$'s) at every step.

Note that \emph{any} physical noise in the system can be 
described in this way.


\pagebreak

%-----------------------------------------------------------------

\begin{center}
\textbf{Results}
\end{center}

\begin{center}
\epsfig{file=../figures/humps.pstex,height=3.5in,width=4.5in}
\end{center}
%{Probability of success -v- time for different noise levels and different numbers
%of qubits.  Black is a run without noise, red is the same run with noise.  Notice that
%where the peaks occur in ``time'' do not change due to noise.  The relative times of the
%peaks are artificial in this graph.}

\begin{center}
\epsfig{file=../figures/max-noise.pstex,height=3.5in,width=4.5in}
\end{center}


\pagebreak

%-----------------------------------------------------------------

\begin{center}
\textbf{Topics for Further Study}
\end{center}

\begin{itemize}
\item What about quantum error correction?
\item What about decoherence--free subspaces?
\item Can Shor be described dynamically?
\end{itemize}

\pagebreak

%-----------------------------------------------------------------

%extra slides
%-----------------------------------------------------------------

\begin{center}
\textbf{How does the QFT find the period?}
\end{center}

\pagebreak

%-----------------------------------------------------------------

\begin{center}
\textbf{What do the geodesic equations look like?}
\end{center}

The geodesic equations of motion come from
\begin{equation*}
\begin{split}
    \ddot{z}^l + \Gamma^l_{ik}\dot{z}^i\dot{z}^k =&\, 0\\
    \ddot{\overline{z}}^l + \Gamma^{\overline{l}}_{\overline{i}\overline{k}}
            \dot{\overline{z}}^i\dot{\overline{z}}^k =&\, 0,
\end{split}
\label{e:geodesics}
\end{equation*}
where the only non-vanishing Christoffel symbols are
\begin{equation*}
\begin{split}
    \Gamma^a_{bc} =& g^{\overline{s}a} g_{c\overline{s},b}\\
    \Gamma^{\overline{a}}_{\overline{b}\overline{c}} =& g^{\overline{a}s} 
                            g_{s\overline{c},b}.
\end{split}
\label{e:xsymb}
\end{equation*}
The geodesic equations are then given in inhomogeneous coordinates by
$\dot{z}=w$ and
\begin{multline*}
    \dot{w}^l = \frac{1}{ \left( 1 + \left| z\right|^2 \right)^2 }
        \biggl\lbrace
            \biggl[ \left( 1 + \left| z\right|^2 \right) 
                    \left( \overline{z}\cdot w \right)
            \biggr] w^l\\
            +
            \biggl[ \left( 1 + \left| z\right|^2 \right)\left( w\cdot w \right)
                    - 2\left( z\cdot w \right)\left( \overline{z}\cdot w \right)
            \biggr] \overline{z}^l\\
            +
            \biggl[ \left( 1 + \left| z\right|^2 \right) 
                    \bigl[ 
                      \left(\overline{z}\cdot w\right)\left(\overline{z}\cdot w\right)
                      +\left(\overline{z}\cdot\overline{z}\right)\left(w\cdot w\right)
                    \bigr]\\
                    - 2\left(\overline{z}\cdot\overline{z}\right) 
                        \left( z\cdot w \right)\left( \overline{z}\cdot w \right)
            \biggr] z^l
        \biggr\rbrace,
\label{e:detailedGeodesics}
\end{multline*}
along with Hermitian conjugates.


\pagebreak



\end{document}
